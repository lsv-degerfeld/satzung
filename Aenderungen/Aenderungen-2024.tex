\documentclass[10pt,a4paper,parskip=half]{scrartcl}
\usepackage[utf8]{inputenc}
\usepackage[ngerman]{babel}
\usepackage[T1]{fontenc}
\usepackage{graphicx}
\usepackage{setspace}
\usepackage{enumitem}
\usepackage[dvipsnames]{xcolor}

\usepackage{pdfpages}

\usepackage{geometry}
\geometry{left=25mm,right=25mm,top=25mm,bottom=45mm}
\usepackage[
automark, % Kapitelangaben in Kopfzeile automatisch erstellen
headsepline, % Trennlinie unter Kopfzeile
ilines % Trennlinie linksbündig ausrichten
]{scrlayer-scrpage }

\useshorthands{+}
\defineshorthand{+S}{\Sentence\ignorespaces}
\defineshorthand{+.}{. \Sentence\ignorespaces}

\pagestyle{scrheadings}
\clearpairofpagestyles

\usepackage{todonotes}

% Kopfzeile
\renewcommand{\headfont}{\normalfont} % Schriftform der Kopfzeile
\ihead{Satzungsänderungen 2024\\\textit{\headmark}}
\chead{\scriptsize{Vorschläge für eine Satzungsänderung im Jahr 2024}\\[1.5ex]}
\ohead{\includegraphics[scale=0.075]{../Logo.png}}
\setlength{\headheight}{20mm} % Höhe der Kopfzeile
\automark{section}

% Fußzeile
\ifoot{\today}
\cfoot{}
\ofoot{\pagemark}
\setlength{\footskip}{20mm}

\frenchspacing % erzeugt ein wenig mehr Platz hinter einem Punkt
% Schusterjungen und Hurenkinder vermeiden
\clubpenalty = 10000
\widowpenalty = 10000
\displaywidowpenalty = 10000
\linespread{1.25} % Mehr Zeilenabstand

\usepackage{scrjura,multicol}
\setlength\columnsep{20pt} % Abstand zwischen den Spalten

\usepackage{helvet}
\addtokomafont{disposition}{\rmfamily} 
\addtokomafont{contract.Clause}{\rmfamily}
\renewcommand{\rmdefault}{phv}
\renewcommand{\familydefault}{\sfdefault}

\usepackage[hidelinks]{hyperref}
\usepackage{changepage}

\title{Satzungsänderungen 2024}
\subtitle{Vorschläge für eine Satzungsänderung des LSV Degerfeld im Jahr 2024}
\author{Patrica Beer, Martin Schick, Florian Lubitz}

\newcommand{\new}[1]{\textcolor{Green}{#1}}
\newcommand{\old}[1]{\textcolor{Red}{#1}}
\newcommand{\change}[1]{
  \begin{adjustwidth}{20pt}{20pt}
    #1
  \end{adjustwidth}
}
\newcommand{\compare}[3]{\change{\subsubsection*{#1}\begin{multicols}{2}#2\columnbreak\\#3\end{multicols}}}


\begin{document}
% \maketitle
\begin{titlepage}

  \begin{center}
    \includegraphics[scale=0.2]{../logo}\\[15ex]

    % \\[10ex]

    \large{Antrag für die nächste Mitgliederversammlung}\\[2ex]

    \LARGE{\textbf{Satzungsänderungen 2024}}\\[2ex]
    \large{Vorschläge für eine Satzungsänderung des LSV Degerfeld im Jahr 2024}\\[30ex]
    \todo[inline]{Entwurf}

    \normalsize

    \textbf{eingereicht am {\today}}\\[1.5ex]
    Patrica Beer\\ Martin Schick\\ Florian Lubitz
    \\[3ex]

  \end{center}

\end{titlepage}


\clearpage


\section*{Vorwort}
Die Satzung des Luftsportverein Degerfeld e.V. wurde zuletzt im Jahr 2018 geändert. Seither haben sich einige Möglichkeiten zur Verbesserung ergeben, die in der Satzung festgehalten werden sollten. Aus diesem Grunde möchten wir einige Änderungen vorschlagen. Die vorgeschlagenen Änderungen, die in diesem Dokument detailliert erläutert werden, sollen auf der anstehenden Mitgliederversammlung im Jahr 2024 zur Abstimmung gestellt werden. Um eine klare Struktur und Übersichtlichkeit zu gewährleisten, sind die vorgeschlagenen Änderungen in mehrere Themenbereiche unterteilt. Diese Themenbereiche reflektieren die verschiedenen Aspekte, die im Zuge der Anpassungen berücksichtigt wurden. Wir möchten darauf hinweisen, dass die meisten Themenbereiche eigenständig funktionieren und unabhängig voneinander betrachtet werden können. Ein paar Bereiche überschneiden sich jedoch und sollten möglichst gemeinsam betrachtet und beschlossen werden. Im Folgenden werden die Änderungen detailliert erläutert und begründet, um eine transparente und nachvollziehbare Diskussion zu ermöglichen.


\tableofcontents
\clearpage

\section*{Art der Gegenüberstellung}

Auf den Folgenden Seiten werden die Änderungen gegenübergestellt, wobei links die bisherige Formulierung und rechts die neue Form dargestellt ist. Die Neuerungen sind in \new{grün} markiert. Die ensprechende bisherige Formulierung ist in \old{rot} markiert. Im Folgenden ein Beispiel, das die Farbgebung verdeutlicht und nicht Teil der Satzungsänderungen ist:

\todo[inline]{Überschriften hinzufügen, z.B. Ausschuss oder "Aufgaben des Ausschuss"}
\compare{§15 Absatz 2 Satz 3}
{\old{Alle} Mitglieder haben das Recht, \old{bei der Jahreshauptversammlung} die Jahresrechnung einzusehen.}
{\new{Die} Mitglieder haben das Recht, \new{nach dem Kassenabschluss} die Jahresrechnung einzusehen.}


\section{Rechtschreibung und kleine Korrekturen ohne inhaltliche Änderung}
In unserer Satzung haben sich im Laufe der Jahre einige Rechtschreibfehler eingeschlichen. Diese sollten korrigiert werden.
Ein Beispiel ist die Schreibweise von "`Auschuß"'. Dieser sollte einheitlich als "`Ausschuss"' geschrieben werden\footnote{https://www.duden.de/node/11150/revision/1394026}.
Außerdem wird die Abkürzung von "`eingetragener Verein"' von "`eV"' auf "`e.V."' geändert\footnote{https://www.duden.de/node/135659/revision/1319174}.

\compare{§1 Absatz 1}
{Der Verein führt den Namen\\>Luftsportverein Degerfeld \old{eV}<}
{Der Verein führt den Namen\\>Luftsportverein Degerfeld \new{e.V.}<}

\compare{§5 Absatz 2}{Über die Aufnahme entscheidet der \old{Ausschuß}.}{Über die Aufnahme entscheidet der \new{Ausschuss}.}

\compare{§6 Absatz 2 Satz 1}
{Über den Ausschluß entscheidet der \old{Ausschuß}.}
{Über den Ausschluß entscheidet der \new{Ausschuss}.}

\compare{§7 Absatz 1 Satz 2}
{In Ausnahmefällen kann der \old{Ausschuß} Ermäßigungen gewähren.}
{In Ausnahmefällen kann der \new{Ausschuss} Ermäßigungen gewähren.}

\compare{§7 Absatz 5}
{Die Start- und Fluggebühren ergeben sich aus den Betriebskosten und werden vom \old{Auschuß} festgesetzt.
  Der \old{Auschuß} ist gehalten,
  diese Gebühren im Interesse der Mitglieder so günstig wie möglich zu halten.}
{Die Start- und Fluggebühren ergeben sich aus den Betriebskosten und werden vom \new{Ausschuss} festgesetzt.
  Der \new{Ausschuss} ist gehalten,
  diese Gebühren im Interesse der Mitglieder so günstig wie möglich zu halten.}

\compare{§9 Absatz 1}{
  Die aktiven ordentlichen Mitglieder gemäß §~4~Punkt 1a und Punkt 1b
  sowie die Ehrenmitglieder der Satzung sind berechtigt,
  die Fluggeräte nach den Beschlüssen des Ausschusses und den Anordnungen des Ausbildungsleiters,
  der Fluglehrer und Flugleiter zu \old{benützen} und an den Veranstaltungen des Vereins teilzunehmen.
}{
  Die aktiven ordentlichen Mitglieder gemäß §~4~Punkt 1a und Punkt 1b
  sowie die Ehrenmitglieder der Satzung sind berechtigt,
  die Fluggeräte nach den Beschlüssen des Ausschusses und den Anordnungen des Ausbildungsleiters,
  der Fluglehrer und Flugleiter zu \new{benutzen} und an den Veranstaltungen des Vereins teilzunehmen.
}

\compare{§10 Absatz 2 Satz 1}{
  Die Mitgliederzusammenkünfte dienen der Aussprache über die den Verein berührenden Fragen,
  wobei sich Vorstand und \old{Ausschuß} über die Meinung der Mitglieder orientieren können.
}{
  Die Mitgliederzusammenkünfte dienen der Aussprache über die den Verein berührenden Fragen,
  wobei sich Vorstand und \new{Ausschuss} über die Meinung der Mitglieder orientieren können.
}

\change{
  \subsubsection*{§11 Absatz 1}\begin{multicols}{2}
    Organe des Vereins sind:
    \begin{enumerate}[noitemsep]
      \item der Vorstand
      \item der \old{Ausschuß}
      \item die Hauptversammlung
    \end{enumerate}
    \columnbreak
    Organe des Vereins sind:
    \begin{enumerate}[noitemsep]
      \item der Vorstand
      \item der \new{Ausschuss}
      \item die Hauptversammlung
    \end{enumerate}
  \end{multicols}
}

\clearpage
\compare{§12 Absatz 4}
{In dringenden Angelegenheiten,
  deren Erledigung nicht bis zu einer \old{Ausschußsitzung} aufgeschoben werden kann,
  entscheidet der Vorstand anstelle des Ausschusses.
  Die Gründe und die Art der Erledigung sind dem \old{Ausschuß} in der nächsten Sitzung mitzuteilen.
}{
  In dringenden Angelegenheiten,
  deren Erledigung nicht bis zu einer \new{Ausschusssitzung} aufgeschoben werden kann,
  entscheidet der Vorstand anstelle des Ausschusses.
  Die Gründe und die Art der Erledigung sind dem \new{Ausschuss} in der nächsten Sitzung mitzuteilen.
}

\compare{§13}
{§ 13 Der \old{Ausschuß}}
{§ 13 Der \new{Ausschuss}}

\compare{§13 Absatz 2 Satz 1}
{Der \old{Ausschuß} und damit auch der Vorstand werden auf zwei Jahre gewählt.}
{Der \new{Ausschuss} und damit auch der Vorstand werden auf zwei Jahre gewählt.}

\compare{§13 Absatz 3 Satz 1}
{Der \old{Ausschuß} ist zuständig für alle Angelegenheiten, welche nicht ausdrücklich dem Vorstand oder der Hauptversammlung vorbehalten sind.}
{Der \new{Ausschuss} ist zuständig für alle Angelegenheiten, welche nicht ausdrücklich dem Vorstand oder der Hauptversammlung vorbehalten sind.}

\compare{§13 Absatz 5}{
  Der \old{Ausschuß} muß einberufen werden,
  wenn es ein Drittel seiner Mitglieder \dots beim Vorstand schriftlich beantragt.
}
{Der \new{Ausschuss} muß einberufen werden,
  wenn es ein Drittel seiner Mitglieder \dots beim Vorstand schriftlich beantragt.}
\compare{§13 Absatz 6 Satz 1}
{Der \old{Ausschuß} beschließt durch Abstimmung.}
{Der \new{Ausschuss} beschließt durch Abstimmung.}

\compare{§13 Absatz 7}
{Der \old{Ausschuß} ist beschlußfähig, wenn einschließlich des Vorstandes mehr als die Hälfte der Mitglieder anwesend sind.}
{Der \new{Ausschuss} ist beschlußfähig, wenn einschließlich des Vorstandes mehr als die Hälfte der Mitglieder anwesend sind.}

\compare{§14 Absatz 3 Satz 2}
{Für die Verabschiedung der Jugendordnung ist der \old{Ausschuß} zuständig.}
{Für die Verabschiedung der Jugendordnung ist der \new{Ausschuss} zuständig.}

\compare{§17 Absatz 1 Satz 4}{Die Zeichnungsberechtigung regelt der \old{Ausschuß}}{Die Zeichnungsberechtigung regelt der \new{Ausschuss}}
\compare{§17 Absatz 1 Satz 5}{Über den Kassenbestand hat er dem Vorstand und dem \old{Ausschuß} auf Verlangen zu berichten und der Hauptversammlung Rechenschaft abzulegen.}{Über den Kassenbestand hat er dem Vorstand und dem \new{Ausschuss} auf Verlangen zu berichten und der Hauptversammlung Rechenschaft abzulegen.}

\compare{§18 Absatz 2}{Sind weitere Luftsportarten eingeführt, so ist die Ausbildung unter sinngemäßer Anwendung des
  Abs. 1 durch den \old{Ausschuß} zu regeln.}{Sind weitere Luftsportarten eingeführt, so ist die Ausbildung unter sinngemäßer Anwendung des
  Abs. 1 durch den \new{Ausschuss} zu regeln.}

\compare{§20 Absatz 1 Satz 1}
{Bau, Unterhaltung und Wartung der Flugzeuge, \dots und Hilfsgeräte geschehen unter Verantwortung \old{des Der/Die technischen Leiter}.}
{Bau, Unterhaltung und Wartung der Flugzeuge, \dots und Hilfsgeräte geschehen unter Verantwortung \new{des/der Technischen Leiter}.}

\compare{§20 Absatz 4 Satz 2}{
  Er meldet Verstöße dem Vorstand, der, notfalls nach Beratung mit dem \old{Ausschuß}, für Abhilfe sorgt.
}{
  Er meldet Verstöße dem Vorstand, der, notfalls nach Beratung mit dem \new{Ausschuss}, für Abhilfe sorgt.
}

\clearpage
\compare{§22 Absatz 2 Satz 1}
{
  Außerordentliche Hauptversammlungen finden statt,
  wenn es der \old{Ausschuß} beschließt oder ein Viertel der ordentlichen und außerordentlichen Mitglieder unter Angabe des Grundes,
  der in den Vereinsaufgaben liegen muß,
  schriftlich beim Vorstand beantragt.
}
{
  Außerordentliche Hauptversammlungen finden statt,
  wenn es der \new{Ausschuss} beschließt oder ein Viertel der ordentlichen und außerordentlichen Mitglieder unter Angabe des Grundes,
  der in den Vereinsaufgaben liegen muß,
  schriftlich beim Vorstand beantragt.
}

\compare{§24 Absatz 2 Satz 2}
{Zum übrigen \old{Ausschuß} kann offen gewählt werden, wenn nur ein Bewerber vorgeschlagen ist und niemand widerspricht.}
{Zum übrigen \new{Ausschuss} kann offen gewählt werden, wenn nur ein Bewerber vorgeschlagen ist und niemand widerspricht.}
\clearpage
\section{Personenbezeichnungen}
Um die Satzung geschlechtsneutral zu gestalten sollen alle Personenbezeichnungen in der Satzung für alle Geschlechter gelten. Dies soll im ersten Paragraphen festgehalten werden.

\compare{§1 Absatz 4}{\em Bisher nicht vorhanden \em}{Sämtliche Personenbezeichnungen gelten für die Geschlechter männlich, weiblich und divers.}

\clearpage
\section{Mitgliedschaft auf Probe}
Bisher mussten sich neue Mitglieder beim Ausschuss vorstellen und dieser hat auf Grund dieser Vorstellung dann um die Aufnahme in den Verein beschlossen. Dieser Vorgang sollte vor allem den Verein vor potentiell Vereinsschädigenden oder andersweitig ungeeigneten Mitgliedern schützen. In der Praxis hat sich gezeigt, dass dieser Vorgang nicht mehr zeitgemäß ist und die Aufnahme in den Verein unnötig verzögert. Deshalb schlagen wir vor die Mitgliedschaft auf Probe einzuführen. Möchte eine Person Ordentliches Mitglied im Verein werden, so kann sie dies bei der Vorstandschaft beantragen. Die Vorstandschaft entscheidet dann inital über die Aufnahme in den Verein. Anschließend wird die Person für ein Jahr als Mitglied auf Probe geführt. Vor Ablauf des Jahres entscheidet der Ausschuss über die Aufnahme in den Verein. Somit haben neue Mitglieder die Möglichkeit, sich in Ruhe in den Verein einzufinden und die Mitglieder des Vereins kennenzulernen und der Verein hat die Möglichkeit das neue Mitglied kennenzulernen und zu entscheiden, ob es in den Verein passt. Die Mitgliedschaft wird automatisch nach einem Jahr in eine ordentliche Mitgliedschaft umgewandelt, wenn der Ausschuss nicht anders entscheidet. Im Vergleich zum bisherigen Verfahren ermöglicht dies eine schnellere Aufnahme und eine bessere Möglichkeit für Mitglied und Verein sich kennenzulernen und zu entscheiden, ob eine Mitgliedschaft im Verein sinnvoll ist. Bei einer fördernden Mitgliedschaft einfällt die Probemitgliedschaft und der Vorstand entscheidet direkt über die Aufnahme.

Diese Regelung ist angelehnt an vergleichbare Regeln in anderen Flugsportvereinen wie beispielsweise dem Luftsportring Aalen e.V.\footnote{https://www.lsr-aalen.de/images/besucher/allgemeines/verein/lsr\_satzung\_2020.pdf}, Aero-Club Rhein-Nahe e.V. oder dem Flugsportverein Speyer \footnote{https://www.flugsportverein-speyer.de/wp-content/uploads/2020/08/Satzung\_2014.pdf}.

\change{
  \subsubsection*{§4 Absatz 1}
  \begin{multicols}{2}
    Mitglieder des Vereins sind:
    \begin{enumerate}[noitemsep]
      \item ordentliche Mitglieder \dots
      \item Fördernde Mitglieder \dots
      \item Ehrenmitglieder \dots
    \end{enumerate}
    \columnbreak
    Mitglieder des Vereins sind:
    \begin{enumerate}[noitemsep]
      \item ordentliche Mitglieder \dots
      \item Fördernde Mitglieder \dots
      \item Ehrenmitglieder \dots
      \item \new{Mitglieder auf Probe}
    \end{enumerate}
  \end{multicols}
}

\compare{§5 Absatz 2}{Über die Aufnahme entscheidet der \old{Ausschuß}.}{Über die Aufnahme entscheidet der \new{Vorstand}.}
\clearpage
\compare{§5 Absatz 4}{\em Bisher nicht vorhanden \em}{Wird eine aktive Mitgliedschaft beantragt,
  so besteht diese zunächst für ein Jahr auf Probe,
  beginnend mit der Annahme des Aufnahmeantrags durch den Vorstand.
  Innerhalb der Probezeit entscheidet der Ausschuss über die endgültige Aufnahme.
  Soweit der Ausschuss keine Entscheidung trifft,
  gilt das Mitglied auf Probe mit Ablauf des Jahres als endgültig aufgenommen.
  Eine Ablehnung über die endgültige Aufnahme wird dem Mitglied auf Probe schriftlich mitgeteilt,
  sie kann ohne Angabe von Gründen erfolgen.
  Hiergegen kann das Mitglied auf Probe innerhalb eines Monats ab Zugang des Ablehnungsschreibens schriftlich Einspruch bei der Mitgliederversammlung einlegen.
  Dieser entscheidet mit einfacher Mehrheit endgültig.
  Bis zum Abschluss dieses vereinsinternen Verfahrens ruhen sämtliche Rechte und Pflichten des betroffenen Mitglieds.
  Bei einer endgültigen Ablehnung wird die entrichtete Aufnahmegebühr in voller Höhe zurückerstattet.
  Die endgültige Ablehnung führt zum Verlust der Mitgliedschaft.}

\change{
  \subsubsection*{§6 Absatz 1}
  \begin{multicols}{2}
    Die Mitgliedschaft erlischt durch
    \begin{enumerate}[label=\alph*)]
      \item Austritt, \dots
      \item Tod
      \item{Ausschluß.} \dots
    \end{enumerate}
    \columnbreak
    Die Mitgliedschaft erlischt durch
    \begin{enumerate}[label=\alph*)]
      \item Austritt, \dots
      \item Tod
      \item{Ausschluß.} \dots
      \item \new{Ablehnung der Dauermitgliedschaft bei Mitgliedschaft auf Probe}
    \end{enumerate}
  \end{multicols}
}

\clearpage
\compare{§8 Absatz 3}
{Aktive Ordentliche Mitglieder gemäß § 4 Punkt 1a und Punkt 1b der Satzung sind verpflichtet, an der Vereinsarbeit teilzunehmen.}
{Aktive Ordentliche Mitglieder sowie Mitglieder auf Probe gemäß § 4 Punkt 1a, Punkt 1b und Punkt 4 der Satzung sind verpflichtet, an der Vereinsarbeit teilzunehmen.}

\compare{§9 Absatz 1}
{Die aktiven ordentlichen Mitglieder gemäß § 4 Punkt 1a und Punkt 1b sowie die Ehrenmitglieder der Satzung sind berechtigt, die Fluggeräte nach den Beschlüssen des Ausschusses und den Anordnungen des Ausbildungsleiters, der Fluglehrer und Flugleiter zu benützen und an den Veranstaltungen des Vereins teilzunehmen.}
{Die aktiven ordentlichen Mitglieder und Mitglieder auf Probe gemäß § 4 Punkt 1a, Punkt 1b und Punkt 4 sowie die Ehrenmitglieder der Satzung sind berechtigt, die Fluggeräte nach den Beschlüssen des Ausschusses und den Anordnungen des Ausbildungsleiters, der Fluglehrer und Flugleiter zu benützen und an den Veranstaltungen des Vereins teilzunehmen.}
\clearpage
\section{Streichung von der Mitgliederliste und Auflösung juristischer Personen}
Bisher lag die Entscheidung über den Ausschluss eines Mitglieds, das mit seinen Zahlungsverpflichtungen im Verzug war, in der Zuständigkeit des Ausschusses. Dieser Prozess ist jedoch recht aufwändig und wird in vielen Vereinen anders gehandhabt. Um eine effizientere und praktikablere Lösung zu schaffen, schlagen wir vor, die Streichung von der Mitgliederliste als zusätzliche Möglichkeit des Erlöschens der Mitgliedschaft in die Satzung aufzunehmen.

Die Streichung von der Mitgliederliste ermöglicht eine schnellere und unkompliziertere Handhabung von Mitgliedschaften, bei denen Zahlungsverpflichtungen über einen längeren Zeitraum nicht erfüllt wurden. Durch diese Maßnahme kann der Verein seine Mitgliederliste auf dem aktuellen Stand halten und gleichzeitig sicherstellen, dass nur diejenigen Mitglieder aktiv bleiben, die ihren Verpflichtungen nachkommen.

Des Weiteren schlagen wir vor, in der Satzung festzuhalten, dass die Mitgliedschaft automatisch erlischt, wenn eine juristische Person aufgelöst wird. Diese Regelung bietet eine klare und automatische Lösung für den Fall, dass eine juristische Person, die Mitglied des Vereins ist, ihre rechtliche Existenz beendet.

\change{
  \subsubsection*{§6 Absatz 1}
  \begin{multicols}{2}
    Die Mitgliedschaft erlischt durch
    \begin{enumerate}[label=\alph*)]
      \item Austritt, \dots
      \item Tod
      \item{Ausschluß.} \dots
    \end{enumerate}
    \columnbreak
    Die Mitgliedschaft erlischt durch
    \begin{enumerate}[label=\alph*)]
      \item Austritt, \dots
      \item Tod, \new{oder Auflösung der juristischen Person}
      \item{Ausschluß.} \dots
      \item \new{Streichung von der Mitgliederliste}
    \end{enumerate}
  \end{multicols}
}
\compare{§6 Absatz 4}{\em Bisher nicht vorhanden \em}{Ein Mitglied kann durch Beschluss des Vorstands von der Mitgliederliste gestrichen werden,
  wenn das Mitglied trotz Mahnung mit der Zahlung von Beiträgen oder sonstigen Zahlungen länger als 6 Monate im Rückstand ist.
  Die Streichung ist dem Mitglied schriftlich mitzuteilen.}

\clearpage
\section{Erhöhung der Freigabebeträge}
In Anbetracht des kontinuierlichen Anstiegs der Kosten für alltägliche Gegenstände sowie für Betriebsmittel, Flugzeug- und Flugplatzunterhaltung in den letzten Jahren, steht der Verein vor der Herausforderung, finanzielle Flexibilität zu bewahren und gleichzeitig effizient handeln zu können. Um diesen Anforderungen gerecht zu werden und sicherzustellen, dass der Verein weiterhin schnell und flexibel agieren kann, schlagen wir vor, die Verfügungsbeträge für den Vorstand und den Ausschuss anzupassen.

Der Vorstand, als maßgebliches Entscheidungsgremium, sollte einen erhöhten Verfügungsbetrag von bis zu 3.000 Euro erhalten. Dies ermöglicht es dem Vorstand, zeitnahe Entscheidungen zu treffen und operative Angelegenheiten ohne unnötige Verzögerungen zu handhaben. Die Erhöhung des Verfügungsbetrags ist insbesondere vor dem Hintergrund der gestiegenen Kosten für alltägliche Ausgaben von großer Bedeutung, um eine effiziente Vereinsführung zu gewährleisten.

Gleichzeitig schlagen wir vor, den Verfügungsbetrag für den Ausschuss auf 20.000 Euro zu erhöhen. Der Ausschuss spielt eine entscheidende Rolle bei der Planung und Umsetzung von langfristigen Projekten sowie bei der Bewältigung finanzieller Herausforderungen. Die Erhöhung des Verfügungsbetrags gibt dem Ausschuss den Spielraum, um notwendige Investitionen zu tätigen und strategische Initiativen zu unterstützen, die zum langfristigen Erfolg des Vereins beitragen.

\compare{§11 Absatz 3}{
  Im Innenverhältnis bedarf der Vorstand für Erwerb und Veräußerung von Vermögensgegenständen über \old{€ 1.500,--} im Einzelfall der vorherigen Zustimmung des Ausschusses.
}{Im Innenverhältnis bedarf der Vorstand für Erwerb und Veräußerung von Vermögensgegenständen über \new{3.000 Euro} im Einzelfall der vorherigen Zustimmung des Ausschusses.}

\compare{§13 Absatz 3 Punkt e)}
{Der Ausschuß ist zuständig für alle Angelegenheiten, welche nicht ausdrücklich dem
  Vorstand oder der Hauptversammlung vorbehalten sind. Dies sind insbesondere\\\dots\\
  e) Vergabe von Aufträgen im Gegenwert von mehr als \old{€ 1.500,--}\\\dots}
{Der Ausschuß ist zuständig für alle Angelegenheiten, welche nicht ausdrücklich dem
  Vorstand oder der Hauptversammlung vorbehalten sind. Dies sind insbesondere\\\dots\\
  e) Vergabe von Aufträgen im Gegenwert von mehr als \new{3.000 Euro}\\\dots}

\clearpage
\compare{§22 Absatz 5 Punkt f)}
{Die Hauptversammlung ist zuständig für: \\ \dots\\
  f) Zustimmung zu Anschaffung und Veräußerung im Einzelfall von Vermögensgegenständen im Wert von über \old{€ 15.ooo,--}, wobei es sich um eine Innenverhältnisregelung handelt\\\dots}
{Die Hauptversammlung ist zuständig für: \\ \dots\\f) Zustimmung zu Anschaffung und Veräußerung im Einzelfall von Vermögensgegenständen im Wert von über \new{20.000 Euro}, wobei es sich um eine Innenverhältnisregelung handelt\\\dots}

\compare{§22 Absatz 5 Punkt g)}
{Die Hauptversammlung ist zuständig für: \\ \dots\\g) Zustimmung zu Schuldenaufnahme im Betrag über \old{€ 15.000,--}, wobei es sich um eine Innenverhältnisregelung handelt\\\dots}
{Die Hauptversammlung ist zuständig für: \\ \dots\\g) Zustimmung zu Schuldenaufnahme im Betrag über \new{20.000 Euro}, wobei es sich um eine Innenverhältnisregelung handelt\\\dots}

\clearpage
\section{Änderung der Zusammensetzung des Vorstands}
\label{sec:vorstand}
In den Anfängen unseres Vereinslebens, geprägt von der Fusion dreier eigenständiger Vereine, wurde eine Vorstandsstruktur etabliert, die den damaligen Gegebenheiten entsprach. Diese Struktur hat uns über die Jahre begleitet, doch inzwischen sind wir nicht mehr die Summe von drei Vereinen, sondern ein gestärkter und geeinter Verein, der sich weiterentwickelt hat.
Die Änderung der Zusammensetzung soll diese historisch geprägte Struktur an die heutigen Gegebenheiten anpassen und den Vorstand handlungsfähiger machen. Eine der wesentlichen Änderungen ist die Einführung des "4 Augen Prinzips" bei der Vertretung nach Außen. Ebenfalls schlagen wir vor, die Festlegung bestimmter Strukturen direkt in der Satzung zu reduzieren und stattdessen eine Geschäftsordnung zu etablieren, dies wird in \autoref{sec:ordnungen} wieder aufgenommen. Zusätzlich schlagen wir vor, bei Bedarf weitere stellvertretende Vorsitzende zu berufen, die sich speziellen Projekten oder Aufgaben widmen können. Dies ermöglicht eine gezieltere Bewältigung von Sonderprojekten wie zum Beispiel der Flurneuordnung, sowie spezialisierte Betreuung von Aufgaben wie der Finanzplanung und -überwachung.

\change{
  \subsubsection*{§12 Absatz 1}
  \begin{multicols}{2}
    Vorstand des Vereins i.S. des §26 BGB sind:
    \begin{itemize}[noitemsep]
      \item der Vorsitzende
      \item der 1. stellvertretende Vorsitzende
      \item \old{der 2. stellvertretende Vorsitzende}
    \end{itemize}
    \old {Sie vertreten den Verein \dots Funktionen allein auszuüben.}

    \columnbreak
    Vorstand des Vereins i.S. des §26 BGB sind:
    \begin{itemize}[noitemsep]
      \item der Vorsitzende
      \item der 1. stellvertretende Vorsitzende
      \item \new{der Kassenverwalter}
      \item \new{der Schriftführer}
      \item \new{maximal zwei weitere stellvertretende Vorsitzende}
    \end{itemize}
    \new{,deren Funktionen im Innenverhältnis in einer vom Vorstand erlassenen Geschäftsordnung zu regeln sind.}
  \end{multicols}
}

Durch Wegfall des bisherigen Absatz 1 Satz 2 und 3 wird dieser in den neuen Absätzen 2 und 3 wieder aufgenommen. Die restlichen Absätze werden entsprechend nach hinten geschoben.

\compare{§12 Absatz 2}
{\em Nach hinten geschoben, wird zu Absatz 4\em}
{Der Verein wird gerichtlich und außergerichtlich von je zwei Vorstandsmitgliedern gemeinsam vertreten.}

\clearpage
\compare{§12 Absatz 3}
{\em Nach hinten geschoben, wird zu Absatz 5\em}
{Der Vorstand führt die laufenden Geschäfte des Vereins, überwacht die Tätigkeiten der übrigen Ausschussmitglieder und vollzieht deren Beschlüße.
  Er leitet die Sitzungen der Mitgliederversammlung und vollzieht deren Beschlüsse. Weiteres regelt die Geschäftsordnung.}

\compare{§12 Absatz 4}
{\em Ehemals Absatz 2\em\\
  Im Innenverhältnis \old{sind die stellvertretenden Vorsitzenden} verpflichtet,
  von \old{ihrem} Vertretungsrecht und von \old{ihrem} Recht auf Ausübung der vorstehend genannten Funktionen nur im Falle der Verhinderung des Vorsitzenden oder mit dessen Ermächtigung Gebrauch zu machen.
  \old{Der 2. stellvertretende Vorsitzende darf die Funktionen nur ausüben,
    wenn sowohl der
    Vorsitzende als auch der 1. stellvertretende Vorsitzende verhindert sind oder mit deren Ermächtigung.}
}
{Im Innenverhältnis \new{ist der stellvertretende Vorsitzende} verpflichtet,
  von \new{seinem} Vertretungsrecht und von \new{seinem} Recht auf Ausübung der vorstehend genannten Funktionen nur im Falle der Verhinderung des Vorsitzenden oder mit dessen Ermächtigung Gebrauch zu machen. \new{Weitere Vertretungsregeln regelt die Geschäftsordnung.}}

\compare{§12 Absatz 7}
{\em Bisher nicht vorhanden\em\\}
{Die zwei weiteren stellvertretenden Vorsitzenden werden bei Bedarf durch den Auschuss für zwei Jahre berufen und gewählt. Sie können jederzeit durch den Ausschuss abberufen werden.}

\compare{§13 Ausschuss, Absatz 2 Satz 1}{
  Der Ausschuß und damit auch der \old{Vorstand} werden auf zwei Jahre gewählt.
}{
  Der Ausschuß und damit auch der \new{Vorsitzende und stellvertretende Vorsitzende} werden auf zwei Jahre gewählt.
}

\clearpage
\change{
  \subsubsection*{§13 Absatz 1 Satz 1}
  \begin{multicols}{2}
    Mitglieder des Ausschusses sind:
    \begin{itemize}[noitemsep]
      \item der Vorsitzende
      \item der 1. stellvertretende Vorsitzende
      \item \old{der 2. stellvertretende Vorsitzende}
      \item der Kassenverwalter
      \item \dots
    \end{itemize}
    \columnbreak
    Mitglieder des Ausschusses sind:
    \begin{itemize}[noitemsep]
      \item der Vorsitzende
      \item der 1. stellvertretende Vorsitzende
      \item \new{bei Bedarf bis zu zwei weitere stellvertretende Vorsitzende}
      \item der Kassenverwalter
      \item \item \dots
    \end{itemize}
  \end{multicols}
}

\compare{§13 Absatz 3 Punkt f)}
{f) Ersatz der während der Amtszeit ausscheidenden Mitglieder des Ausschusses (ohne \old{Vorstand}) bis zur nächsten Hauptversammlung.}
{f) Ersatz der während der Amtszeit ausscheidenden Mitglieder des Ausschusses (ohne \new{Vorsitzenden und 1.stellvertretendem Vorsitzenden}) bis zur nächsten Hauptversammlung.}


\clearpage
\section{Verkleinerung des Ausschusses um die Beisitzer und Nachbesetzung}
Die Größe unseres Ausschusses ist über die Jahre stetig gewachsen und umfasst im Moment 15 Personen. Diese Anzahl an Personen ist für die Entscheidungsfindung im Verein nicht unbedingt notwendig und unter Umständen schwer zu koordinieren. Insbesondere das Finden von Terminen für Ausschusssitzungen zeigt sich als problembehaftet. Des öfteren ist der Auschuss nicht beschlussfähig, da unzureichend Mitglieder in der Sitzung anwesend sind. Dies führt zu Verzögerungen bei Entscheidungen und kann unter Umständen zu Problemen führen, wenn Entscheidungen nicht rechtzeitig getroffen werden können. Um die Effizienz des Ausschusses zu erhöhen und die Entscheidungsfindung zu beschleunigen, schlagen wir vor, die Anzahl der Ausschussmitglieder zu reduzieren. Dazu sollen die Positionen der Beisitzer aufgelöst werden.

\change{
  \subsubsection*{§13 Absatz 1 Satz 1}
  \begin{multicols}{2}
    Mitglieder des Ausschusses sind:
    \begin{itemize}[noitemsep]
      \item der Vorsitzende
      \item der 1. stellvertretende Vorsitzende
      \item der 2. stellvertretende Vorsitzende
      \item der Kassenverwalter
      \item der Ausbildungsleiter
      \item der/die Technischer Leiter
      \item der Flugbetriebsleiter
      \item der Platz- und Hallenverwalter
      \item der Jugendleiter
      \item der Schriftführer
      \item der Pressewart
      \item \old{4 Beisitzer}
    \end{itemize}
    \columnbreak
    Mitglieder des Ausschusses sind:
    \begin{itemize}[noitemsep]
      \item der Vorsitzende
      \item der 1. stellvertretende Vorsitzende
      \item der 2. stellvertretende Vorsitzende
      \item der Kassenverwalter
      \item der Ausbildungsleiter
      \item der/die Technischer Leiter
      \item der Flugbetriebsleiter
      \item der Platz- und Hallenverwalter
      \item der Jugendleiter
      \item der Schriftführer
      \item der Pressewart
    \end{itemize}
  \end{multicols}
}

Um die Nachbesetzung für ausgeschiedene Ausschussmitglieder besser zu definieren soll der entsprechende Absatz etwas angepasst werden und auch die Definition der Wahlperiode soll verdeutlicht werden.

\compare{§ 13 Absatz 2}{Der Ausschuß und damit auch der Vorstand werden \old{auf zwei Jahre} gewählt. Wiederwahl ist zulässig.}{Der Ausschuss und damit auch der Vorstand werden \new{von der Mitgliederversammlung auf die Dauer von zwei Jahren,
    vom Tage der Wahl an gerechnet,} gewählt. \new{Er bleibt jedoch bis zur Neuwahl des Ausschusses beziehungsweise des Vorstandes im Amt.}
  Wiederwahl ist zulässig.}

\compare{§ 13 Absatz 3 Punkt f)}{Ersatz der während der Amtszeit ausscheidenden Mitglieder des Ausschusses (ohne Vorstand) bis zur nächsten Hauptversammlung}{Ersatz der während der Amtszeit ausscheidenden Mitglieder des Ausschusses (ohne Vorstand) bis zur nächsten Hauptversammlung \new{durch Zuwahl.}}

\compare{§13 Absatz 3 Punkt g)}
{\em Bisher nicht vorhanden \em}
{g) Überwachung der Tätigkeiten des Vorstandes}

\compare{§13 Absatz 7}
{der Ausschuß ist beschlußfähig, wenn \old{einschließlich des Vorstandes} mehr als die Hälfte der Mitglieder anwesend sind.}
{Der Ausschuss ist beschlußfähig,
  wenn \new{ein Vorsitzender oder Stellvertreter und} mehr als die Hälfte der \new{restlichen} Mitglieder anwesend sind.}

\clearpage
\section{Anpassung der zu entlastenden Ämter}
Da nur der Vorstand nach §26 BGB die Rechts und Geschäftsfähigkeit des Vereins besitzt, ist es nicht sinnvoll, den Ausschuss zu entlasten. Mit der Änderung in \autoref*{sec:vorstand} ist damit auch der Kassier inbegriffen und somit alle zu entlastenden Ämter enthalten. Deshalb soll die Entlastung des Ausschusses aus der Satzung gestrichen werden. Hierzu wird in §22 Absatz 5 ein Punkt nach a) eingefügt und die bisherigen Punkte b) bis j) werden um einen Buchstaben nach hinten verschoben.

\compare{§22 Absatz 5 Punkt a)}
{Die Hauptversammlung ist zuständig für: \\a) Entgegennahme der Berichte des Ausschusses \old{und dessen Entlastung}\\
  \old{b)} Wahl des Vorstandes und der übrigen Ausschussmitglieder sowie der beiden Kassenprüfer, mit Ausnahme des Jugendleiters.\\\dots}
{Die Hauptversammlung ist zuständig für: \\a) Entgegennahme der Berichte des Ausschusses\\
  \new{b) Entlastung des vertretungsberechtigten Vorstandes nach §26 BGB (§ 12 Absatz 1 Satz 1)}\\
  \new{c)} Wahl des Vorstandes und \dots \\\dots}
\clearpage
\section{Änderung des Zeitpunktes der Mitgliederversammlung}
Gemäß der aktuellen Satzung ist die Hauptversammlung im ersten Quartal des Jahres anberaumt. Bislang stand unser Kassier jedoch vor der Herausforderung, den Rechnungsabschluss nach Erhalt der Daten vom Steuerberater unter erheblichem Zeitdruck zu erstellen. Diese zeitliche Enge hatte leider in der Vergangenheit negative Auswirkungen auf die Qualität des Kassenbericht. Zusätzlich wurde im bereits laufenden Jahr der Haushaltsplan vorgestellt und verabschiedet.

Um diesem Zeitdruck entgegenzuwirken und eine qualitativ hochwertige Arbeit zu gewährleisten, schlagen wir vor, die Satzung dahingehend zu ändern, dass eine Hauptversammlung künftig mindestens einmal jährlich stattfindet, idealerweise im Oktober. Während dieser Versammlung sollen der Rechnungsabschluss des Vorjahres und der Haushaltsplan für das kommende Jahr präsentiert werden. Diese zeitliche Anpassung ermöglicht ausreichend Zeit für die Erstellung des Kassenbericht, einschließlich der notwendigen Zeit für die Kassenprüfer. Zudem erlaubt sie eine frühzeitige Beschlussfassung über den Haushaltsplan für das kommende Jahr noch im Vorjahr.

Die Änderung der Satzung zur Verlegung der Hauptversammlung auf einmal jährlich, vorzugsweise im Oktober, bringt nicht nur eine zeitliche Entlastung für den Kassier, sondern bietet auch den Vorteil, dass bei einer Änderung im Vorstand oder Ausschuss mehr Zeit für Übergabe und Einarbeitung zum Saisonende zur Verfügung steht, im Gegensatz zur bisherigen Regelung zum Saisonstart.

\compare{§22 Absatz 1}{
  Die ordentliche Hauptversammlung (Jahreshauptversammlung) findet \old{alljährlich im ersten Kalendervierteljahr} statt.
  Der Vorstand hat die Einladung unter Angabe von Zeitpunkt,
  Tagungsort und Tagesordnung allen Mitgliedern spätestens \old{einen Monat} zuvor durch einfachen Brief oder per E-Mail zuzustellen.
}{
  Die ordentliche Hauptversammlung (Jahreshauptversammlung) findet \new{einmal jährlich} statt.
  Der Vorstand hat die Einladung unter Angabe von Zeitpunkt,
  Tagungsort und Tagesordnung allen Mitgliedern spätestens \new{drei Wochen} zuvor durch einfachen Brief oder per E-Mail zuzustellen.
  \new{Das Einladungsschreiben gilt als dem Mitglied zugegangen,wenn es an die letzte vom Mitglied dem Verein schriftlich bekannt gegebene Adresse gerichtet ist.}
}

\compare{§22 Absatz 3}{Anträge der Mitglieder können in der Haupt\-ver\-samm\-lung nur behandelt werden,
  wenn sie spätestens \old{bis zum 31. Januar} schriftlich beim Vorsitzenden mit Begründung eingegangen sind.}
{Anträge der Mitglieder können in der \new{ordentlichen} Haupt\-ver\-samm\-lung nur behandelt werden,
  wenn sie spätestens \new{6 Wochen vor der Hauptversammlung} beim Vorsitzenden mit Begründung eingegangen sind.
  \new{Der Vorsitzende hat diese Frist zwei Wochen vor Ablauf derselben den Mitgliedern mitzuteilen.}
}

\compare{§13 Absatz 2}
{Der Ausschuß und damit auch der Vorstand werden \old{auf zwei Jahre gewählt.}
  Wiederwahl ist zulässig.}
{Der Ausschuss und damit auch der Vorstand werden \new{von der Mitgliederversammlung auf die Dauer von zwei Jahren, vom Tage der Wahl an gerechnet, gewählt.
    Er bleibt jedoch bis zur Neuwahl des Ausschusses beziehungsweise des Vorstandes im Amt}.  Wiederwahl ist zulässig.}

\clearpage
\section{Aufnahme von Ordnungen in die Satzung}
\label{sec:ordnungen}
Um die Vereinsangelegenheiten und internen Abläufe zu regeln, ist es sinnvoll, Ordnungen zu erlassen. Viele Vereine nutzen Ordnungen, um die Vereinsarbeit zu strukturieren und regeln. Der Vorteil von Ordnungen ist, dass sie im Gegensatz zur Satzung leichter zu ändern sind und somit flexibler auf aktuelle Gegebenheiten angepasst werden können. Bisher definieren wir Ordnungen in der Satzung nicht explizit, nutzen diese jedoch in abgeänderter Form, wie zum Beispiel für die Fluggebühren und Arbeitsstunden. Um diese Ordnungen in die Satzung aufzunehmen und die Möglichkeit zu schaffen, weitere Ordnungen zu erlassen, schlagen wir vor, einen Paragraphen in die Satzung aufzunehmen, der die Erlassung von Ordnungen regelt.

\change{
  \subsubsection*{§ 27 Ordnungen}
  Zur Durchführung der Satzung kann sich der Verein folgende Ordnungen geben:
  \begin{enumerate}[label=\alph*),noitemsep]
    \item Geschäftsordnung
    \item Finanzordnung
    \item Beitragsordnung
    \item Datenschutzordnung
    \item Jugendordnung
    \item Ehrungsordnung
  \end{enumerate}
  sowie weitere, dem Verein dienliche Ordnungen.
  \todo[inline]{Straf- oder Disziplinarordnung}

  Die Mitgliederversammlung ist für den Erlass der Ordnungen zuständig.
  Ausgenommen davon sind die Geschäftsordnung,
  die vom Vorstand zu beschließen ist,
  sowie die Jugendordnung,
  die von der Vereinsjugend zu beschließen und vom Ausschuss zu bestätigen ist.

}

\compare{§6 Absatz 1 Punkt c)}
{Ausschluß. Ausgeschlossen kann werden, wer gegen die Vorschriften, die der Flugsicherheit die-
  nen, gegen die Satzung, die Beschlüsse und Anordnungen der Organe und das Ansehen des Vereins wiederholt oder erheblich verstoßen hat, sich beharrlich ohne ausreichenden Grund weigert, die ihm von den Organen des Vereins übertragenen Aufgaben zu erfüllenoder mit
  Zahlungsverpflichtungen trotz schriftlicher Mahnung länger als ein Jahr im Rückstand ist.}
{Ausschluß. Ausgeschlossen kann werden, wer gegen die Vorschriften, die der Flugsicherheit die-
  nen, gegen die Satzung, \new{die Ordnungen,} die Beschlüsse und Anordnungen der Organe und das
  Ansehen des Vereins wiederholt oder erheblich verstoßen hat, sich beharrlich ohne ausreichenden
  Grund weigert, die ihm von den Organen des Vereins übertragenen Aufgaben zu erfüllen oder mit
  Zahlungsverpflichtungen trotz schriftlicher Mahnung länger als ein Jahr im Rückstand ist.}

\clearpage
\section{Hinzufügen von Strafbestimmungen}
Das Rechtsverständnis, in dem wir uns bewegen bedeutet: Keine Strafe ohne Regel/ Gesetzt. Soll der Verein oder dessen Organe in der Lage sein, dass Fehlverhalten sanktioniert werden soll, müssen die Mitglieder dem Verein/ Organ des Vereins eine Grundlage hierfür einräumen. Ohne diese klare Regelung in der Satzung ist es einem Organ des Vereins unmöglich, rechtssicher eine Maßnahmen gegenüber einem Mitglied zu beschließen. Die Folge daraus ist, dass ggf. keine Maßnahme ergriffen wird. Entweder folgt das Gremium dem Grundsatz, keine Strafe ohne Regel. Oder das Gremium scheut sich im Hinblick auf einen möglichen Rechtsstreit, eine Maßnahme zu ergreifen. Ohne eine solche Bestimmung bleibt es letztlich der Leidensfähigkeit der Mitglieder der Vereinsorgane überlassen, ob Maßnahmen ergriffen werden.

Zur Aufnahme der Strafbestimmungen soll folgender Paragraph in die Satzung aufgenommen werden.

\change{
  \subsubsection*{§ 28 Strafbestimmungen}
  Sämtliche Mitglieder des Vereins unterliegen der Ordnungsgewalt des Vereins.
  Der Vorstand kann gegen Mitglieder,
  die gegen die Satzung oder gegen Beschlüsse der Organe verstoßen oder das Ansehen,
  die Ehre und das Vermögen des Vereines schädigen,
  folgende Maßnahmen verhängen:
  \begin{enumerate}[label=\alph*),noitemsep]
    \item Rüge, Ermahnung, Verwarung, Verweis
    \item Zeitlich begrenztes Verbot der Teilnahme am Flugbetrieb und an Veranstaltungen des Vereines
    \item Geldstrafe bis zu € 250,00 je Einzelfall
    \item Ausschluss gemäß § 6 Punkt c)
    \item Zeitlich begrenztes Verbot des Betretens der vereinseigenen Anlagen und Einrichtungen
  \end{enumerate}
}
\clearpage
\section{Hinzufügen einer Datenschutzordnung}
Insbesondere seit Inkrafttreten der Datenschutzgrundverordnung (DSGVO) im Jahr 2018 ist der Datenschutz auch ein wichtiges Thema für Vereine. Unter anderem sind diese verpflichtet, die Einhaltung der DSGVO sicherzustellen und die Grundzüge der Datenerhebung und -verarbeitung schriftlich festzuhalten. Zu diesem Zwecke ist eine grundlegende Definition in der Satzung sinnvoll sowie die Erstellung einer Datenschutzordnung, die weitere Einzelheiten regelt. Die Verankerung des Datenschutzes in der Satzung ist ein wichtiger Schritt,  um die Einhaltung der DSGVO sicherzustellen und gibt über die weitergehende Datenschutzordnung zudem die Möglichkeit, die Datenschutzbestimmungen bei Bedarf (z.B. Adaption einer anderen Vereinsverwaltung oder geänderte Datenangaben) zu ändern.

Um diesen Ansprüchen gerecht zu werden soll ein Paragraph in die Satzung aufgenommen werden, der die grundlegenden Belange des Datenschutzes regelt und die Einhaltung der Datenschutzgrundverordnung sicherstellt.

\change{
  \subsection*{§ 29 Datenschutz}

  Unter Beachtung der gesetzlichen Vorgaben und Bestimmungen der EU\--Daten\-schutz\--Grund\-verordnung (DSGVO) und des Bundes\-datenschutz\-gesetzes (BDSG) werden zur Erfüllung der Zwecke und Aufgaben des Vereins personenbezogene Daten über persönliche und sachliche Verhältnisse der Mitglieder des Vereins erhoben und in den vereinseigenen EDV-Systemen gespeichert, genutzt und verarbeitet.

  Als Mitglied des Württembergischen Landessportbundes e.V. (WLSB) ist der Verein verpflichtet, seine Mitglieder an den WLSB sowie an den jeweiligen Fachverband zu melden.

  Der Verein erlässt eine Datenschutzordnung, in der weitere Einzelheiten der Datenerhebung und der Datenverwendung sowie technische und organisatorische Maßnahmen zum Schutz der Daten aufgeführt sind. Die Datenschutzordnung wird auf Vorschlag der Vorstandschaft durch die Mitgliederversammlung beschlossen.

  Um die Aktualität der gemäß Absatz 1 erfassten Daten zu gewährleisten,
  sind die Mitglieder verpflichtet, Veränderungen umgehend dem Verein mitzuteilen.
}
\clearpage

\section{Einführung eines Ehrenrates}
Der Ehrenrat soll als unabhängiges Gremium ins Leben gerufen, um die Integrität, Fairness und Ehre innerhalb des Vereins zu wahren. Seine Mitglieder werden aufgrund ihrer langjährigen Bindung und ihres Engagements für den Verein ausgewählt, wodurch sie ein tiefes Verständnis für die Werte und Notwendigkeiten des Vereins mitbringen.

Die Voraussetzungen für die Mitgliedschaft im Ehrenrat, wie das Erreichen des vierzigsten Lebensjahres und mindestens zehn Jahre ununterbrochene aktive Mitgliedschaft, wurden bewusst festgelegt, um sicherzustellen, dass die gewählten Mitglieder über reichhaltige Erfahrungen und eine solide Verbundenheit mit dem Verein verfügen. Die Bedingung, nicht dem Vorstand anzugehören, gewährleistet zudem eine unvoreingenommene Betrachtung von Angelegenheiten, die den Verein betreffen.

Besonders bedeutend ist auch die Funktion des Ehrenrats bei der Durchführung von Schlichtungserfahren. Diese Rolle unterstützt die Aufrechterhaltung eines harmonischen Umfelds innerhalb des Vereins und gewährleistet, dass Konflikte fair und gerecht gelöst werden.
Hierzu soll folgender Paragraph in die Satzung aufgenommen werden.
\change{
  \subsubsection*{§ 30 Ehrenrat}
  Der Ehrenrat besteht aus fünf Mitgliedern, die vom Ausschuss alle zwei Jahre gewählt werden.

  Wählbar sind nur Mitglieder, die das vierzigste Lebensjahr vollendet haben, mindestens zehn Jahre als aktives Mitglied dem Verein und nicht dem Vorstand angehören.

  Der Ehrenrat wählt seinen Vorsitzenden selbst.

  Die Aufgaben des Ehrenrates sind:

  \begin{enumerate}[label=\alph*),noitemsep]
    \item Unterbreitung von Ehrungsvorschlägen an den Vorstand gemäß den Bestimmungen der Ehrenordnung
    \item Unterbreitung von Änderungsvorschlägen zur Ehrenordnung an den Ausschuss
    \item Durchführung von Schlichtungs- und Ausschlussverfahren
  \end{enumerate}

  Der Ehrenrat tritt nur bei Bedarf zusammen.
}

Die Wahl des Ehrenrates soll auch in die Aufgaben des Ausschusses mit aufgenommen werden.

\compare{§13 Absatz 3 Punkt g)}{
  \em Bisher nicht vorhanden \em
} {
  g) Wahl des Ehrenrates
}


\section{Anhang}
\begin{enumerate}
  \item Darstellung der Unterschiede der Änderung in der Satzung
  \item Satzung in der neuen Fassung
\end{enumerate}

\includepdf[pages=-]{../diff.pdf}
\includepdf[pages=-]{../Satzung.pdf}

\end{document}

