\documentclass[10pt,a4paper,parskip=half]{scrartcl}
%DIF LATEXDIFF DIFFERENCE FILE
%DIF DEL Satzung-old.tex   Thu Nov 23 12:34:20 2023
%DIF ADD Satzung.tex       Fri Feb 16 15:06:30 2024
\usepackage[utf8]{inputenc}
\usepackage[ngerman]{babel}
\usepackage[T1]{fontenc}
\usepackage{graphicx}
\usepackage{setspace}
\usepackage{enumitem}

\usepackage{geometry}
\geometry{a4paper,left=25mm,right=25mm,top=25mm,bottom=45mm}
\usepackage[
automark, % Kapitelangaben in Kopfzeile automatisch erstellen
headsepline, % Trennlinie unter Kopfzeile
ilines % Trennlinie linksbündig ausrichten
]{scrlayer-scrpage }

\useshorthands{+}
\defineshorthand{+S}{\Sentence\ignorespaces}
\defineshorthand{+.}{. \Sentence\ignorespaces}

\pagestyle{scrheadings}
\clearpairofpagestyles

% Kopfzeile
\renewcommand{\headfont}{\normalfont} % Schriftform der Kopfzeile
%DIF 26c26
%DIF < \ihead{Satzung des Luftsportverein Degerfeld eV\\\textit{\headmark}}
%DIF -------
\ihead{Satzung des Luftsportverein Degerfeld e.V.\\\textit{\headmark}} %DIF > 
%DIF -------
\chead{\\[1ex]\scriptsize{17. Oktober 2020}}
\ohead{\includegraphics[scale=0.075]{Logo.png}}
\setlength{\headheight}{20mm} % Höhe der Kopfzeile

% Fußzeile
\ifoot{\today}
\cfoot{}
\ofoot{\pagemark}
\setlength{\footskip}{20mm}

\frenchspacing % erzeugt ein wenig mehr Platz hinter einem Punkt
% Schusterjungen und Hurenkinder vermeiden
\clubpenalty = 10000
\widowpenalty = 10000
\displaywidowpenalty = 10000
\linespread{1.25} % Mehr Zeilenabstand

\usepackage{scrjura,multicol}
\setlength\columnsep{20pt} % Abstand zwischen den Spalten

\usepackage{helvet}
\addtokomafont{disposition}{\rmfamily} 
\addtokomafont{contract.Clause}{\rmfamily}
\renewcommand{\rmdefault}{phv}

\usepackage[hidelinks]{hyperref}


\title{Satzung}
%DIF < \subtitle{nach der Satzungsänderung in der Mitgliederversammlung vom 17. Oktober 2020}
%DIF -------
\subtitle{nach der Satzungsänderung in der Mitgliederversammlung vom XX. XXXXXX XXXX} %DIF > 
%DIF PREAMBLE EXTENSION ADDED BY LATEXDIFF
%DIF UNDERLINE PREAMBLE %DIF PREAMBLE
\RequirePackage[normalem]{ulem} %DIF PREAMBLE
\RequirePackage{color}\definecolor{RED}{rgb}{1,0,0}\definecolor{BLUE}{rgb}{0,0,1} %DIF PREAMBLE
\providecommand{\DIFaddtex}[1]{{\protect\color{blue}\uwave{#1}}} %DIF PREAMBLE
\providecommand{\DIFdeltex}[1]{{\protect\color{red}\sout{#1}}}                      %DIF PREAMBLE
%DIF SAFE PREAMBLE %DIF PREAMBLE
\providecommand{\DIFaddbegin}{} %DIF PREAMBLE
\providecommand{\DIFaddend}{} %DIF PREAMBLE
\providecommand{\DIFdelbegin}{} %DIF PREAMBLE
\providecommand{\DIFdelend}{} %DIF PREAMBLE
\providecommand{\DIFmodbegin}{} %DIF PREAMBLE
\providecommand{\DIFmodend}{} %DIF PREAMBLE
%DIF FLOATSAFE PREAMBLE %DIF PREAMBLE
\providecommand{\DIFaddFL}[1]{\DIFadd{#1}} %DIF PREAMBLE
\providecommand{\DIFdelFL}[1]{\DIFdel{#1}} %DIF PREAMBLE
\providecommand{\DIFaddbeginFL}{} %DIF PREAMBLE
\providecommand{\DIFaddendFL}{} %DIF PREAMBLE
\providecommand{\DIFdelbeginFL}{} %DIF PREAMBLE
\providecommand{\DIFdelendFL}{} %DIF PREAMBLE
%DIF HYPERREF PREAMBLE %DIF PREAMBLE
\providecommand{\DIFadd}[1]{\texorpdfstring{\DIFaddtex{#1}}{#1}} %DIF PREAMBLE
\providecommand{\DIFdel}[1]{\texorpdfstring{\DIFdeltex{#1}}{}} %DIF PREAMBLE
\newcommand{\DIFscaledelfig}{0.5}
%DIF HIGHLIGHTGRAPHICS PREAMBLE %DIF PREAMBLE
\RequirePackage{settobox} %DIF PREAMBLE
\RequirePackage{letltxmacro} %DIF PREAMBLE
\newsavebox{\DIFdelgraphicsbox} %DIF PREAMBLE
\newlength{\DIFdelgraphicswidth} %DIF PREAMBLE
\newlength{\DIFdelgraphicsheight} %DIF PREAMBLE
% store original definition of \includegraphics %DIF PREAMBLE
\LetLtxMacro{\DIFOincludegraphics}{\includegraphics} %DIF PREAMBLE
\newcommand{\DIFaddincludegraphics}[2][]{{\color{blue}\fbox{\DIFOincludegraphics[#1]{#2}}}} %DIF PREAMBLE
\newcommand{\DIFdelincludegraphics}[2][]{% %DIF PREAMBLE
\sbox{\DIFdelgraphicsbox}{\DIFOincludegraphics[#1]{#2}}% %DIF PREAMBLE
\settoboxwidth{\DIFdelgraphicswidth}{\DIFdelgraphicsbox} %DIF PREAMBLE
\settoboxtotalheight{\DIFdelgraphicsheight}{\DIFdelgraphicsbox} %DIF PREAMBLE
\scalebox{\DIFscaledelfig}{% %DIF PREAMBLE
\parbox[b]{\DIFdelgraphicswidth}{\usebox{\DIFdelgraphicsbox}\\[-\baselineskip] \rule{\DIFdelgraphicswidth}{0em}}\llap{\resizebox{\DIFdelgraphicswidth}{\DIFdelgraphicsheight}{% %DIF PREAMBLE
\setlength{\unitlength}{\DIFdelgraphicswidth}% %DIF PREAMBLE
\begin{picture}(1,1)% %DIF PREAMBLE
\thicklines\linethickness{2pt} %DIF PREAMBLE
{\color[rgb]{1,0,0}\put(0,0){\framebox(1,1){}}}% %DIF PREAMBLE
{\color[rgb]{1,0,0}\put(0,0){\line( 1,1){1}}}% %DIF PREAMBLE
{\color[rgb]{1,0,0}\put(0,1){\line(1,-1){1}}}% %DIF PREAMBLE
\end{picture}% %DIF PREAMBLE
}\hspace*{3pt}}} %DIF PREAMBLE
} %DIF PREAMBLE
\LetLtxMacro{\DIFOaddbegin}{\DIFaddbegin} %DIF PREAMBLE
\LetLtxMacro{\DIFOaddend}{\DIFaddend} %DIF PREAMBLE
\LetLtxMacro{\DIFOdelbegin}{\DIFdelbegin} %DIF PREAMBLE
\LetLtxMacro{\DIFOdelend}{\DIFdelend} %DIF PREAMBLE
\DeclareRobustCommand{\DIFaddbegin}{\DIFOaddbegin \let\includegraphics\DIFaddincludegraphics} %DIF PREAMBLE
\DeclareRobustCommand{\DIFaddend}{\DIFOaddend \let\includegraphics\DIFOincludegraphics} %DIF PREAMBLE
\DeclareRobustCommand{\DIFdelbegin}{\DIFOdelbegin \let\includegraphics\DIFdelincludegraphics} %DIF PREAMBLE
\DeclareRobustCommand{\DIFdelend}{\DIFOaddend \let\includegraphics\DIFOincludegraphics} %DIF PREAMBLE
\LetLtxMacro{\DIFOaddbeginFL}{\DIFaddbeginFL} %DIF PREAMBLE
\LetLtxMacro{\DIFOaddendFL}{\DIFaddendFL} %DIF PREAMBLE
\LetLtxMacro{\DIFOdelbeginFL}{\DIFdelbeginFL} %DIF PREAMBLE
\LetLtxMacro{\DIFOdelendFL}{\DIFdelendFL} %DIF PREAMBLE
\DeclareRobustCommand{\DIFaddbeginFL}{\DIFOaddbeginFL \let\includegraphics\DIFaddincludegraphics} %DIF PREAMBLE
\DeclareRobustCommand{\DIFaddendFL}{\DIFOaddendFL \let\includegraphics\DIFOincludegraphics} %DIF PREAMBLE
\DeclareRobustCommand{\DIFdelbeginFL}{\DIFOdelbeginFL \let\includegraphics\DIFdelincludegraphics} %DIF PREAMBLE
\DeclareRobustCommand{\DIFdelendFL}{\DIFOaddendFL \let\includegraphics\DIFOincludegraphics} %DIF PREAMBLE
%DIF COLORLISTINGS PREAMBLE %DIF PREAMBLE
\RequirePackage{listings} %DIF PREAMBLE
\RequirePackage{color} %DIF PREAMBLE
\lstdefinelanguage{DIFcode}{ %DIF PREAMBLE
%DIF DIFCODE_UNDERLINE %DIF PREAMBLE
  moredelim=[il][\color{red}\sout]{\%DIF\ <\ }, %DIF PREAMBLE
  moredelim=[il][\color{blue}\uwave]{\%DIF\ >\ } %DIF PREAMBLE
} %DIF PREAMBLE
\lstdefinestyle{DIFverbatimstyle}{ %DIF PREAMBLE
	language=DIFcode, %DIF PREAMBLE
	basicstyle=\ttfamily, %DIF PREAMBLE
	columns=fullflexible, %DIF PREAMBLE
	keepspaces=true %DIF PREAMBLE
} %DIF PREAMBLE
\lstnewenvironment{DIFverbatim}{\lstset{style=DIFverbatimstyle}}{} %DIF PREAMBLE
\lstnewenvironment{DIFverbatim*}{\lstset{style=DIFverbatimstyle,showspaces=true}}{} %DIF PREAMBLE
%DIF END PREAMBLE EXTENSION ADDED BY LATEXDIFF

\begin{document}

\thispagestyle{plain}
\begin{center}
  \includegraphics[scale=0.2]{Logo.png}\\[5ex]

  \Huge{\textbf{Satzung}}\\[1.5ex]
  \large{nach der Satzungsänderung in der Mitgliederversammlung vom}\\[1.5ex]

  \normalsize

  \textbf{\DIFdelbegin %DIFDELCMD < \Large{17. Oktober 2020}%%%
\DIFdelend \DIFaddbegin \Large{XX. XXXXXX XXXX}\DIFaddend }\\

\end{center}

\begin{contract}
  % \begin{multicols}{2}

    \Clause{title={Name, Sitz und Rechtsform}}
    Der Verein führt den Namen\\
    >Luftsportverein Degerfeld \DIFdelbegin \DIFdel{eV}\DIFdelend \DIFaddbegin \DIFadd{e.V.}\DIFaddend <

    Er hat seinen Sitz in Albstadt.

    Er ist in das Vereinsregister beim Amtsgericht Stuttgart unter der Nr. VR 400196 eingetragen

    \DIFaddbegin \DIFadd{Sämtliche Personenbezeichnungen gelten für die Geschlechter männlich, weiblich und divers.
    }

    \DIFaddend \SubClause{title={Mitgliedschaft in Verbänden}}
    Der Verein ist Mitglied im Ba\-den-Würt\-tem\-ber\-gischen Luftfahrt\-verband (BWLV),
    im Hans-Kellner-Gedächtnis\-fonds (HKF),
    im Deutschen Aero Club (DAeC) und im Würt\-tem\-ber\-gischen Landessportbund (WLSB).
    Der Verein und seine Mitglieder anerkennen als für sich rechtsverbindlich die Satzungsbestimmungen und Ordnungen des Baden-Württembergischen Luftfahrt\-verbandes und im Würt\-tem\-ber\-gischen Landes\-sport\-bundes in ihrer jeweils gültigen Fassung.

    \Clause{title={Zweck, Gemeinnützigkeit und Mittelverwendung}}
    Der Verein verfolgt ausschließlich und unmittelbar gemeinnützige Zwecke im Sinne des Abschnittes "`Steuerbegünstigte Zwecke"' der Abgabenordnung.

    Zweck des Vereins ist die Pflege und die Förderung des Luftsports.
    Er fördert alle damit zusammenhängenden Angelegenheiten,
    insbesondere durch den Bau,
    Kauf und Unterhaltung von Luftfahrtgeräten,
    Hilfsgeräten und Luftfahrteinrichtungen für sportliche Zwecke und die Ausbildung seiner Mitglieder.
    Eines seiner Hauptanliegen ist die Betreuung und Förderung der Jugend.
    Er darf keine anderen Ziele verfolgen und sich nicht parteipolitisch oder militärähnlich betätigen.

    Der Verein ist selbstlos tätig,
    er verfolgt nicht in erster Linie eigenwirtschaftliche Zwecke.
    Mittel des Vereins dürfen nur für die satzungsmäßigen Zwecke des Vereins verwendet werden.
    Es darf keine Person durch Ausgaben,
    die dem Zweck der Körperschaft fremd sind oder durch unverhältnismäßig hohe Vergütungen begünstigt werden.

    Die Hauptversammlung bestimmt,
    welche Luftsportarten betrieben werden.

    \Clause{title={Geschäftsjahr}}
    Geschäftsjahr ist das Kalenderjahr.

    \Clause{title={Mitglieder}}
    \label{C:Mitglieder}

    Mitglieder des Vereins sind:
    \begin{enumerate}
      \item ordentliche Mitglieder
            \begin{enumerate}
              \item Erwachsene Mitglieder,
                    die das 25.Lebensjahr vollendet haben,
                    die den Luftsport im Verein aktiv
                    ausüben. \label{S:OrdentlicheMitglieder:Erwachsene}
              \item Jugendliche Mitglieder,
                    die das 25. Lebensjahr noch nicht vollendet haben,
                    die den Luftsport im Verein aktiv ausüben. \label{S:OrdentlicheMitglieder:Jugendliche}
              \item{passive Mitglieder} Erwachsene und Jugendliche Mitglieder,
                    die im Verein den Luftsport aktiv ausgeübt haben,
                    derzeit aber den Luftsport nicht aktiv ausüben.
            \end{enumerate}
      \item{Fördernde Mitglieder} Fördernde Mitglieder sind natürliche oder juristische Personen,
            die an den Zielen des Vereins interessiert sind,
            den Luftsport im Verein nicht aktiv ausüben oder ausgeübt haben.

      \item{Ehrenmitglieder}\\
            Natürliche Personen,
            die sich um den Verein besonders verdient gemacht haben,
            können vom Ausschuss zu Ehrenmitgliedern ernannt werden.
            Vorsitzende und stellvertretende Vorsitzende können nach Beendigung ihres Amtes zu Ehrenvorsitzenden ernannt werden,
            wenn sie sich um den Verein außerordentlich verdient gemacht haben.
      \DIFaddbegin \item \DIFadd{Mitglieder auf Probe}\label{S:MitgliederAufProbe}
    \DIFaddend \end{enumerate}\label{S:OrdentlicheMitglieder}

    \Clause{title={Erwerb der Mitgliedschaft}}

    Die Aufnahme ist unter Anerkennung der Satzung und Entrichtung des Aufnahmebeitrages schriftlich zu beantragen.
    Bewerber für den aktiven Luftsport haben außerdem eine Enthaftungserklärung abzugeben und ihre Fliegertauglichkeit nachzuweisen.
    Noch nicht volljährige Mitglieder benötigen die Zustimmung ihrer gesetzlichen Vertreter.

    Über die Aufnahme entscheidet der \DIFdelbegin \DIFdel{Ausschuß}\DIFdelend \DIFaddbegin \DIFadd{Vorstand}\DIFaddend .

    Die Ablehnung des Aufnahmeantrages braucht nicht begründet zu werden.
    Sie ist vom Vorsitzenden dem Bewerber schriftlich mitzuteilen.
    Der Aufnahmebeitrag ist zurückzuzahlen.

    \DIFaddbegin \DIFadd{Wird eine aktive Mitgliedschaft beantragt,
    so besteht diese zunächst für ein Jahr auf Probe,
    beginnend mit der Annahme des Aufnahmeantrags durch den Vorstand.
    Innerhalb der Probezeit entscheidet der Ausschuss über die endgültige Aufnahme.
    Soweit der Ausschuss keine Entscheidung trifft,
    gilt das Mitglied auf Probe mit Ablauf des Jahres als endgültig aufgenommen.
    Eine Ablehnung über die endgültige Aufnahme wird dem Mitglied auf Probe schriftlich mitgeteilt,
    sie kann ohne Angabe von Gründen erfolgen.
    Hiergegen kann das Mitglied auf Probe innerhalb eines Monats ab Zugang des Ablehnungsschreibens schriftlich Einspruch bei der Mitgliederversammlung einlegen.
    Diese entscheidet mit einfacher Mehrheit endgültig.
    Bis zum Abschluss dieses vereinsinternen Verfahrens ruhen sämtliche Rechte und Pflichten des betroffenen Mitglieds.
    Bei einer endgültigen Ablehnung wird die entrichtete Aufnahmegebühr in voller Höhe zurückerstattet.
    Die endgültige Ablehnung führt zum Verlust der Mitgliedschaft.
    }

    \DIFaddend \Clause{title={Erlöschen der Mitgliedschaft}}
    \label{C:ErloeschenDerMitgliedschaft}
    Die Mitgliedschaft erlischt durch
    \begin{enumerate}[label=\alph*)]
      \item Austritt,
            der nur zum Schluß des Geschäftsjahres möglich ist.
            Die Kündigung muß dem Vorsitzenden bis zum 30. November schriftlich gegen Empfangsnachweis zugegangen sein.
      \item Tod \DIFdelbegin \DIFdel{.
            }%DIFDELCMD < 

%DIFDELCMD <       %%%
\DIFdelend \DIFaddbegin \DIFadd{oder Auflösung der juristischen Person
      }\DIFaddend \item{Ausschluß.} Ausgeschlossen kann werden,
            wer gegen die Vorschriften,
            die der Flugsicherheit dienen,
            gegen die Satzung,
            die \DIFaddbegin \DIFadd{Ordnungen,
            die }\DIFaddend Beschlüsse und Anordnungen der Organe und das Ansehen des Vereins wiederholt oder erheblich verstoßen hat,
            sich beharrlich ohne ausreichenden Grund weigert,
            die ihm von den Organen des Vereins übertragenen Aufgaben zu erfüllen\DIFdelbegin \DIFdel{oder mit Zahlungsverpflichtungen trotz schriftlicher Mahnung länger als ein Jahr im Rückstand ist}\DIFdelend .\label{S:Ausschluss}
      \DIFaddbegin \item \DIFadd{Ablehnung der Dauermitgliedschaft bei Mitgliedschaft auf Probe
      }\item \DIFadd{Streichung von der Mitgliederliste,            
    }\DIFaddend \end{enumerate}

    Über den Ausschluß entscheidet der \DIFdelbegin \DIFdel{Ausschuß}\DIFdelend \DIFaddbegin \DIFadd{Ausschuss}\DIFaddend .
    Der Beschluß ist zu begründen und dem Betroffennen vom Vorsitzenden schriftlich gegen Empfangsnachweis mitzuteilen.
    Gegen den Ausschluß kann binnen eines Monats nach Zustellung schriftlich Beschwerde an die
    Hauptversammlung erhoben werden,
    die endgültig entscheidet.
    Hierauf ist im Beschluß hinzuweisen.

    Ausscheidende Mitglieder oder deren Erben haben keinen Anspruch an das Vereinsvermögen.
    Verpflichtungen gegenüber dem Verein bleiben unberührt.
    Der Mitgliedsausweis ist zurückzugeben.

    \DIFaddbegin \DIFadd{Ein Mitglied kann durch Beschluss des Vorstands von der Mitgliederliste gestrichen werden,
    wenn das Mitglied trotz Mahnung mit der Zahlung von Beiträgen oder sonstigen Zahlungen länger als 6 Monate im Rückstand ist.
    Die Streichung ist dem Mitglied schriftlich mitzuteilen.
    }

    \DIFaddend \Clause{title={Beiträge}}
    Die Vereinsbeiträge der Mitglieder,
    die Aufnahmebeiträge und die Sonderzahlungen bestimmt die Hauptversammlung.
    In Ausnahmefällen kann der \DIFdelbegin \DIFdel{Ausschuß }\DIFdelend \DIFaddbegin \DIFadd{Ausschuss }\DIFaddend Ermäßigungen gewähren.

    Die Beiträge und sonstigen Zahlungen an den BWLV,
    DAeC,
    HKF und den WLSB werden den Mitgliedsbeiträgen zugerechnet und bedürfen nicht der Zustimmung der Hauptversammlung.

    Die Beiträge sind im 1. Quartal eines Jahres in einer Summe zu bezahlen,
    in der Regel durch Lastschrifteinzug.

    Ehrenmitglieder sind von Pflichtbeiträgen befreit.

    Die Start- und Fluggebühren ergeben sich aus den Betriebskosten und werden vom \DIFdelbegin \DIFdel{Ausschuß }\DIFdelend \DIFaddbegin \DIFadd{Ausschuss }\DIFaddend festgesetzt.
    Der \DIFdelbegin \DIFdel{Ausschuß }\DIFdelend \DIFaddbegin \DIFadd{Ausschuss }\DIFaddend ist gehalten,
    diese Gebühren im Interesse der Mitglieder so günstig wie möglich zu halten.

    \DIFaddbegin \DIFadd{Die Mitglieder sind verpflichtet, bei Bedarf des Vereines Arbeitsleistungen zu erbringen.
    Die Anzahl der jährlichen Arbeitsstunden beschließt die Mitgliederversammlung.
    Nicht erbrachte Arbeitsstunden müssen durch die Leistung eines Geldbetrages abgegolten werden.
    Die Höhe dieses Geldbetrages pro nicht geleisteter Arbeitsstunde beschließt die Mitgliederversammlung.
    Einzelheiten regelt die Beitragsordnung.
    }

    \DIFaddend \Clause{title={Pflichten der Mitglieder}}

    Jedes Mitglied hat sich den Bestimmungen der Satzung und den Beschlüssen der Organe des Vereins zu fügen und diese auszuführen.
    Im Arbeitseinsatz ist den Anordnungen des Technischen Leiters und der Warte sowie des Platz- und Hallenverwalters,
    im Flugdienst den Anordnungen des Flugleiters,
    der Fluglehrer und des Startleiters Folge zu leisten.
    Die den Luftsportverkehr regelnden Bestimmungen sind gewissenhaft zu befolgen.
    \DIFdelbegin %DIFDELCMD < 

%DIFDELCMD <     %%%
\DIFdelend Das Vereinseigentum und die sonst zur Verfügung stehenden Gegenstände sind schonend zu behandeln.
    Beschädigungen sind unverzüglich dem Technischen Leiter bzw dem Platz-und Hallenverwalter zu melden.
    Für vorsätzliche und grob fahrlässige Beschädigungen haftet das Mitglied in voller Höhe.

    Alle Tätigkeiten im Verein sind ehrenamtlich.
    Die notwendigen Auslagen können auf Beschluß des Vorstandes ersetzt werden.

    Aktive Ordentliche Mitglieder \DIFaddbegin \DIFadd{sowie Mitglieder auf Probe }\DIFaddend gemäß \refL{C:Mitglieder}~\autoref{S:OrdentlicheMitglieder:Erwachsene}\DIFdelbegin \DIFdel{und }\DIFdelend \DIFaddbegin \DIFadd{, }\DIFaddend \autoref{S:OrdentlicheMitglieder:Jugendliche} \DIFaddbegin \DIFadd{und }\autoref{S:MitgliederAufProbe} \DIFaddend der Satzung sind verpflichtet,
    an der Vereinsarbeit teilzunehmen.

    \Clause{title={Rechte der Mitglieder}}

    Die aktiven ordentlichen Mitglieder \DIFaddbegin \DIFadd{und Mitglieder auf Probe }\DIFaddend gemäß \refL{C:Mitglieder}~\autoref{S:OrdentlicheMitglieder:Erwachsene}\DIFdelbegin \DIFdel{und }\DIFdelend \DIFaddbegin \DIFadd{, }\DIFaddend \autoref{S:OrdentlicheMitglieder:Jugendliche} \DIFaddbegin \DIFadd{und }\autoref{S:MitgliederAufProbe} 
    \DIFaddend sowie die Ehrenmitglieder der Satzung sind berechtigt,
    die Fluggeräte nach den Beschlüssen des Ausschusses und den Anordnungen des Ausbildungsleiters,
    der Fluglehrer und Flugleiter zu \DIFdelbegin \DIFdel{benützen }\DIFdelend \DIFaddbegin \DIFadd{benutzen }\DIFaddend und an den Veranstaltungen des Vereins teilzunehmen.

    Stimm- und wahlberechtigt sind die ordentlichen Mitglieder und die Ehrenmitglieder

    Solange Mitglieder mit Zahlungen trotz Mahnung im Rückstand sind,
    ruhen ihre Rechte.

    \Clause{title={Mitgliederzusammenkunft}}
    Unter Vorsitz eines Ausschussmitgliedes findet in der Regel monatlich an einem im Voraus bestimmten Tag an einem längere Zeit gleichbleibenden Ort eine Mitgliederzusammenkunft statt.

    Die Mitgliederzusammenkünfte dienen der Aussprache über die den Verein berührenden Fragen,
    wobei sich Vorstand und \DIFdelbegin \DIFdel{Ausschuß }\DIFdelend \DIFaddbegin \DIFadd{Ausschuss }\DIFaddend über die Meinung der Mitglieder orientieren können.
    Darüberhinaus dient die Mitgliederzusammenkunft der Förderung der Kameradschaft und der theoretischen Weiterbildung der Mitglieder.

    \Clause{title={Organe des Vereins}}
    Organe des Vereins sind:
    \begin{enumerate}[noitemsep]
      \item der Vorstand
      \item der \DIFdelbegin \DIFdel{Ausschuß
      }\DIFdelend \DIFaddbegin \DIFadd{Ausschuss
      }\DIFaddend \item die Hauptversammlung
    \end{enumerate}

    \Clause{title={Der Vorstand}}
    \DIFaddbegin \label{C:Vorstand}
    \DIFaddend Vorstand des Vereins i.S. des §26 BGB sind:
    \begin{itemize}[noitemsep]
      \item der Vorsitzende
      \item der 1. stellvertretende Vorsitzende
      \item der \DIFdelbegin \DIFdel{2. stellvertretende Vorsitzende
    .
    }\DIFdelend \DIFaddbegin \DIFadd{Kassenverwalter
      }\item \DIFadd{der Schriftführer
      }\item \DIFadd{maximal zwei weitere stellvertretende Vorsitzende
    }\DIFaddend \end{itemize}\DIFdelbegin \DIFdel{Sie vertreten den Verein }\DIFdelend \DIFaddbegin \label{S:Vorstandsmitglieder}
    \DIFadd{,deren Funktionen im Innenverhältnis in einer vom Vorstand erlassenen Geschäftsordnung zu regeln sind.
    }

    \DIFadd{Der Verein wird }\DIFaddend gerichtlich und außergerichtlich \DIFdelbegin \DIFdel{,
    führen }\DIFdelend \DIFaddbegin \DIFadd{von je zwei Vorstandsmitgliedern gemeinsam vertreten.
    }

    \DIFadd{Der Vorstand führt }\DIFaddend die laufenden Geschäfte des Vereins, \DIFdelbegin \DIFdel{überwachen }\DIFdelend \DIFaddbegin \DIFadd{überwacht }\DIFaddend die Tätigkeiten der übrigen \DIFdelbegin \DIFdel{Ausschußmitglieder,
    leiten die Sitzungen des Ausschusses und der Hauptversammlung und vollziehen deren Beschlüße.
    Jeder von ihnen ist allein vertretungsberechtigt und befugt,
    auch die vorstehend genannten Funktionen allein auszuüben}\DIFdelend \DIFaddbegin \DIFadd{Ausschussmitglieder und vollzieht deren Beschlüße. 
    Er leitet die Sitzungen der Mitgliederversammlung und vollzieht deren Beschlüsse.
    Weiteres regelt die Geschäftsordnung}\DIFaddend .

    Im Innenverhältnis \DIFdelbegin \DIFdel{sind die stellvertretenden Vorsitzenden }\DIFdelend \DIFaddbegin \DIFadd{ist der stellvertretende Vorsitzende }\DIFaddend verpflichtet,
    von \DIFdelbegin \DIFdel{ihrem }\DIFdelend \DIFaddbegin \DIFadd{seinem }\DIFaddend Vertretungsrecht und von \DIFdelbegin \DIFdel{ihrem }\DIFdelend \DIFaddbegin \DIFadd{seinem }\DIFaddend Recht auf Ausübung der vorstehend genannten Funktionen nur im Falle der Verhinderung des Vorsitzenden oder mit dessen Ermächtigung Gebrauch zu machen. \DIFdelbegin \DIFdel{Der 2. stellvertretende Vorsitzende darf die Funktionen nur ausüben,
    wenn sowohl der
    Vorsitzende als auch der 1. stellvertretende Vorsitzende verhindert sind oder mit deren Ermächtigung}\DIFdelend \DIFaddbegin \DIFadd{Weitere Vertretungsregeln regelt die Geschäftsordnung}\DIFaddend .

    Im Innenverhältnis bedarf der Vorstand für Erwerb und Veräußerung von Vermögensgegenständen über \DIFdelbegin \DIFdel{€ 1.500,-- }\DIFdelend \DIFaddbegin \DIFadd{3.000 Euro }\DIFaddend im Einzelfall der vorherigen Zustimmung des Ausschusses.

    In dringenden Angelegenheiten,
    deren Erledigung nicht bis zu einer \DIFdelbegin \DIFdel{Ausschußsitzung }\DIFdelend \DIFaddbegin \DIFadd{Ausschusssitzung }\DIFaddend aufgeschoben werden kann,
    entscheidet der Vorstand anstelle des Ausschusses.
    Die Gründe und die Art der Erledigung sind dem \DIFdelbegin \DIFdel{Ausschuß }\DIFdelend \DIFaddbegin \DIFadd{Ausschuss }\DIFaddend in der nächsten Sitzung mitzuteilen.

    \DIFdelbegin %DIFDELCMD < \Clause{title={Der Ausschuß}}
%DIFDELCMD <     %%%
\DIFdelend \DIFaddbegin \DIFadd{Die zwei weiteren stellvertretenden Vorsitzenden werden bei Bedarf durch den Auschuss für zwei Jahre berufen und gewählt.
    Sie können jederzeit durch den Ausschuss abberufen werden.
    }

    \Clause{title={Der Ausschuss}}
    \label{C:Ausschuss}
    \DIFaddend Mitglieder des Ausschusses sind:
    \begin{itemize}[noitemsep]
      \item der Vorsitzende
      \item der 1. \DIFdelbegin \DIFdel{stellvertretendeVorsitzende
      }\DIFdelend \DIFaddbegin \DIFadd{stellvertretende Vorsitzende
      }\DIFaddend \item \DIFdelbegin \DIFdel{der 2. }\DIFdelend \DIFaddbegin \DIFadd{bei Bedarf bis zu zwei weitere }\DIFaddend stellvertretende Vorsitzende
      \item der Kassenverwalter
      \item der Ausbildungsleiter
      \item der/die \DIFdelbegin \DIFdel{Technischer }\DIFdelend \DIFaddbegin \DIFadd{Technische/n }\DIFaddend Leiter
      \item der Flugbetriebsleiter
      \item der Platz- und Hallenverwalter
      \item der Jugendleiter
      \item der Schriftführer
      \item der Pressewart
    \DIFdelbegin %DIFDELCMD < \item %%%
\item%DIFAUXCMD
\DIFdel{4 Beisitzer
    }\DIFdelend \end{itemize}
    Mit Ausnahme der Vorsitzenden und des Kassenverwalters können mehrere Aufgaben von derselben Person übernommen werden.

    Der \DIFdelbegin \DIFdel{Ausschuß }\DIFdelend \DIFaddbegin \DIFadd{Ausschuss }\DIFaddend und damit auch der \DIFdelbegin \DIFdel{Vorstand werden auf zwei Jahre }\DIFdelend \DIFaddbegin \DIFadd{Vorsitzende und stellvertretende Vorsitzende werden von der Mitgliederversammlung auf die Dauer von zwei Jahren,
    vom Tage der Wahl an gerechnet, }\DIFaddend gewählt.
    \DIFaddbegin \DIFadd{Er bleibt jedoch bis zur Neuwahl des Ausschusses beziehungsweise des Vorsitzenden und stellvertretende Vorsitzenden im Amt.
    }\DIFaddend Wiederwahl ist zulässig.

    Der \DIFdelbegin \DIFdel{Ausschuß }\DIFdelend \DIFaddbegin \DIFadd{Ausschuss }\DIFaddend ist zuständig für alle Angelegenheiten,
    welche nicht ausdrücklich dem Vorstand oder der Hauptversammlung vorbehalten sind.
    Dies sind insbesondere
    \begin{enumerate}[label=\alph*),noitemsep]
      \item Aufstellung der vereinsinternen Regeln für Flugbetrieb,
            Ausbildung sowie Platz- und Geräteunterhaltung
      \item Zustimmung zur Durchführung von Veranstaltungen und Festsetzung der dafür notwendigen Bestimmungen
      \item Anordnung von Maßnahmen gegen Mitglieder
      \item Vorbereitung der Hauptversammlung
      \item Vergabe von Aufträgen im Gegenwert von mehr als \DIFdelbegin \DIFdel{€ 1.500,--
      }\DIFdelend \DIFaddbegin \DIFadd{3.000 Euro
      }\DIFaddend \item Ersatz der während der Amtszeit ausscheidenden Mitglieder des Ausschusses (ohne \DIFdelbegin \DIFdel{Vorstand}\DIFdelend \DIFaddbegin \DIFadd{Vorsitzenden und 1.stellvertretendem Vorsitzenden}\DIFaddend ) bis zur nächsten Hauptversammlung \DIFdelbegin \DIFdel{.
    }\DIFdelend \DIFaddbegin \DIFadd{durch Zuwahl.
      }\item \DIFadd{Wahl des Ehrenrates
      }\item \DIFadd{Überwachung der Tätigkeiten des Vorstandes
    }\DIFaddend \end{enumerate}

    Zu den Sitzungen des Ausschusses ist vom Vorstand mit angemessener Frist unter Angabe von Zeitpunkt,
    Tagungsort und Tagesordnung schriftlich einzuladen.
    In außerordentlichen Fällen kann die Einladung kurzfristig und formlos erfolgen.

    Der \DIFdelbegin \DIFdel{Ausschuß }\DIFdelend \DIFaddbegin \DIFadd{Ausschuss }\DIFaddend muß einberufen werden,
    wenn es ein Drittel seiner Mitglieder unter Angabe des Grundes,
    der in den Aufgaben des Vereins liegen muß,
    beim Vorstand schriftlich beantragt.

    Der \DIFdelbegin \DIFdel{Ausschuß }\DIFdelend \DIFaddbegin \DIFadd{Ausschuss }\DIFaddend beschließt durch Abstimmung.
    Bei Stimmengleichheit entscheidet die Stimme des Vorsitzenden.

    Der \DIFdelbegin \DIFdel{Ausschuß }\DIFdelend \DIFaddbegin \DIFadd{Ausschuss }\DIFaddend ist beschlußfähig,
    wenn \DIFdelbegin \DIFdel{einschließlich des Vorstandes }\DIFdelend \DIFaddbegin \DIFadd{ein Vorsitzender oder Stellvertreter und }\DIFaddend mehr als die Hälfte der \DIFaddbegin \DIFadd{restlichen }\DIFaddend Mitglieder anwesend sind.

    \Clause{title={Die Luftsportjugend}}
    Der Luftsportverein Degerfeld e.V. unterhält im Rahmen ihrer Jugendarbeit die Jugendorganisation Luftsportjugend Degerfeld.

    Die Luftsportjugend Degerfeld arbeitet gemäß der verabschiedeten Jugendordnung.

    Die Jugendordnung wird von der Luftsportjugend beschlossen. Für die Verabschiedung der Jugendordnung ist der \DIFdelbegin \DIFdel{Ausschuß }\DIFdelend \DIFaddbegin \DIFadd{Ausschuss }\DIFaddend zuständig.

    \Clause{title={Der Schriftführer}}
    Der Schriftführer führt die Protokolle über die Sitzungen des Ausschusses und der Hauptversammlung.
    Die Protokolle sind vom Sitzungsleiter und vom Schriftführer zu unterschreiben.
    Sie sollen den wesentlichen Teil der Beratungen,
    die gefaßten Beschlüsse und das Stimmenverhältnis bei Wahlen und Abstimmungen enthalten.

    Die Mitglieder können die Protokolle einsehen,
    soweit nicht persönliche Belange Anderer berührt sind.

    \Clause{title={Der Pressewart}}
    Der Pressewart unterrichtet die örtliche Presse und die Fachblätter übergeordneter Organisationen über die Tätigkeit des Vereins und wirbt durch Veröffentlichung von Berichten für den Luftsport im allgemeinen und für den Verein im Besonderen.
    Er unterrichtet die Mitglieder über allgemein interessierende Vereinsangelegenheiten.
    Alle Mitglieder sind gehalten, ihm Material und Lichtbilder über besondere Ereignisse zur Verfügung zu stellen.

    \Clause{title={Der Kassenverwalter}}
    Der Kassenverwalter verwaltet die Kasse und führt die Rechung.
    Er hat für den Einzug der Einnahmen zu sorgen.
    Er leistet Zahlung auf Anweisung des Vorstandes.
    Die Zeichnungsberechtigung regelt der \DIFdelbegin \DIFdel{Ausschuß}\DIFdelend \DIFaddbegin \DIFadd{Ausschuss}\DIFaddend .
    Über den Kassenbestand hat er dem Vorstand und dem \DIFdelbegin \DIFdel{Ausschuß }\DIFdelend \DIFaddbegin \DIFadd{Ausschuss }\DIFaddend auf Verlangen zu berichten und der Hauptversammlung Rechenschaft abzulegen.
    Die Mitglieder haben das Recht,
    bei der Jahreshauptversammlung die Jahresrechnung einzusehen.

    Der Kassenverwalter führt ein fortlaufendes Mitgliederverzeichnis,
    aus welchem neben Namen,
    Vornamen,
    Geburtstag,
    Wohnung und Beruf,
    der Ein- und Austritt und die Art der Mitgliedschaft ersichtlich sind.
    Er führt ein Inventarverzeichnis über das Vereinsvermögen und eine Spenderliste.

    Rückständige Zahlungsverpflichtungen sind nach angemessener Frist schriftlich anzumahnen.
    Erfolgt innerhalb eines Monats keine Zahlung,
    ist dem Vorsitzenden zu berichten.

    \Clause{title={Der Ausbildungsleiter}}
    Dem Ausbildungsleiter sind alle Fluglehrer und Startwindenfahrer des Vereins unterstellt.
    Er teilt die Auszubildenden den Fluglehrern zu,
    überwacht die Ausbildung und den Leistungsstand der Fluglehrer und aller Piloten.
    Er erteilt im Benehmen mit dem zuständigen Fluglehrer die Flugberechtigung und kann sie widerrufen.
    Dabeisind die Regeln des Ausschusses zu beachten.

    Sind weitere Luftsportarten eingeführt,
    so ist die Ausbildung unter sinngemäßer Anwendung des Abs. 1 durch den \DIFdelbegin \DIFdel{Ausschuß }\DIFdelend \DIFaddbegin \DIFadd{Ausschuss }\DIFaddend zu regeln.

    \Clause{title={Der Flugbetriebsleiter}}
    Der Flugbetriebsleiter ist zuständig für die ordentliche und sichere Durchführung des Flugbetriebes im Rahmen der gesetzlichen Bestimmungen und behördlicher Anordnungen.

    \DIFdelbegin %DIFDELCMD < \Clause{title={Der/Die technischen Leiter}}
%DIFDELCMD <     %%%
\DIFdelend \DIFaddbegin \Clause{title={Der/Die Technische/n Leiter}}
    \DIFaddend Bau,
    Unterhaltung und Wartung der Flugzeuge,
    Startwinden,
    Fahrzeuge,
    Nachrichtenmittel und Hilfsgeräte geschehen unter Verantwortung des\DIFdelbegin \DIFdel{Der}\DIFdelend /\DIFdelbegin \DIFdel{Die technischen }\DIFdelend \DIFaddbegin \DIFadd{der Technischen }\DIFaddend Leiter.
    Bei Bedarf können für diese Aufgabe mehrere Personen bestellt werden.

    Für die einzelnen Gerätearten sind Warte zu bestellen,
    denen die notwendigen Hilfskräfte zuzuteilen sind.

    Der/Die \DIFdelbegin \DIFdel{technischen }\DIFdelend \DIFaddbegin \DIFadd{Technische/n }\DIFaddend Leiter ist für die Einsatzbereitschaft jedes Gerätes verantwortlich,
    er veranlaßt die Beschaffung von Ersatzteilen und Betriebsstoffen und die Ausführung von Instandsetzungen und Unterhaltsarbeiten.

    Der/Die \DIFdelbegin \DIFdel{technischen }\DIFdelend \DIFaddbegin \DIFadd{Technische/n }\DIFaddend Leiter ist weisungsbefugt gegenüber Warten und Hilfskräften.
    Er meldet Verstöße dem Vorstand,
    der,
    notfalls nach Beratung mit dem \DIFdelbegin \DIFdel{Ausschuß}\DIFdelend \DIFaddbegin \DIFadd{Ausschuss}\DIFaddend ,
    für Abhilfe sorgt.

    \Clause{title={Der Platz- und Hallenverwalter}}
    Die Instandhaltung des Landeplatzes Albstadt-Degerfeld und seiner Einrichtungen ist Aufgabe des Platz- und Hallenverwalters.
    Er sorgt insbesondere für Ordnung und Sauberkeit auf dem Platz,
    in den Hallen und auf dem Vorplatz,
    den Wegen und Parkplätzen.
    Die notwendigen Hilfskräfte sind ihm auf Anforderung zur Verfügung zu stellen.

    \Clause{title={Die Hauptversammlung}}
    \label{C:Hauptversammlung}
    Die ordentliche Hauptversammlung (Jahreshauptversammlung) findet \DIFdelbegin \DIFdel{alljährlich im ersten Kalendervierteljahr }\DIFdelend \DIFaddbegin \DIFadd{einmal jährlich }\DIFaddend statt.
    Der Vorstand hat die Einladung unter Angabe von Zeitpunkt,
    Tagungsort und Tagesordnung allen Mitgliedern spätestens \DIFdelbegin \DIFdel{einen Monat }\DIFdelend \DIFaddbegin \DIFadd{drei Wochen }\DIFaddend zuvor durch einfachen Brief oder per E-Mail zuzustellen.
    \DIFaddbegin \DIFadd{Das Einladungsschreiben gilt als dem Mitglied zugegangen,
    wenn es an die letzte vom Mitglied dem Verein schriftlich bekannt gegebene Adresse gerichtet ist.
    }\DIFaddend 

    Außerordentliche Hauptversammlungen finden statt,
    wenn es der \DIFdelbegin \DIFdel{Ausschuß }\DIFdelend \DIFaddbegin \DIFadd{Ausschuss }\DIFaddend beschließt oder ein Viertel der ordentlichen und außerordentlichen Mitglieder unter Angabe des Grundes,
    der in den Vereinsaufgaben liegen muß,
    schriftlich beim Vorstand beantragt.
    Bei unabweisbarem Bedarf kann die Einberufungsfrist bis auf drei Tage verkürzt werden.
    Im Übrigen gelten die für die ordentliche Hauptversammlung getroffenen Regelungen entsprechend.

    Anträge der Mitglieder können in der \DIFaddbegin \DIFadd{ordentlichen }\DIFaddend Haupt\-ver\-samm\-lung nur behandelt werden,
    wenn sie spätestens \DIFdelbegin \DIFdel{bis zum 31. Januar schriftlich }\DIFdelend \DIFaddbegin \DIFadd{6 Wochen vor der Hauptversammlung }\DIFaddend beim Vorsitzenden mit Begründung eingegangen sind.
    \DIFaddbegin \DIFadd{Der Vorsitzende hat diese Frist zwei Wochen vor Ablauf derselben den Mitgliedern mitzuteilen.
    }\DIFaddend 

    Die Hauptversammlung ist ohne Rücksicht auf die Zahl der erschienenen Mitglieder beschlußfähig,
    wenn sie satzungsgemäß einberufen ist.
    Hierüber ist vom Vorstand zu Beginn der Versammlung Feststellung zu treffen.

    Die Hauptversammlung ist zuständig für
    \begin{enumerate}[label=\alph*),noitemsep]
      \item Entgegennahme der Berichte des Ausschusses
      \DIFdelbegin \DIFdel{und dessen Entlastung }\DIFdelend \DIFaddbegin \item \DIFadd{Entlastung des vertretungsberechtigten Vorstandes nach §26 BGB (\ref{S:Vorstandsmitglieder})
      }\DIFaddend \item Wahl des Vorstandes und der übrigen Ausschussmitglieder sowie der beiden Kassenprüfer,
            mit Ausnahme des Jugendleiters.
      \item Festsetzung der Aufnahmebeiträge,
            der Vereinsbeiträge und der Sonderzahlungen
      \item Entscheidung über Beschwerden gegen Beschlüsse des Ausschusses
      \item Aufstellung des Haushaltsplanes
      \item Zustimmung zu Anschaffung und Veräußerung im Einzelfall von Vermögensgegenständen im Wert von über \DIFdelbegin \DIFdel{€ 15.ooo,
            --,
            }\DIFdelend \DIFaddbegin \DIFadd{20.000 Euro,
            }\DIFaddend wobei es sich um eine Innenverhältnisregelung handelt
      \item Zustimmung zu Schuldenaufnahme im Betrag über \DIFdelbegin \DIFdel{€ 15.000,
            --,
            }\DIFdelend \DIFaddbegin \DIFadd{20.000 Euro,
            }\DIFaddend wobei es sich um eine Innenverhältnisregelung handelt
      \item Aufnahme und Beendigung einzelner Sportarten
      \item Änderung der Satzung
      \item Auflösung des Vereins
    \end{enumerate}

    \Clause{title={Die Kassenprüfer}}
    Die Kassenprüfer dürfen nicht Mitglieder des Ausschusses sein.

    Den Kassenprüfern ist die Jahresrechnung nebst allen Belegen mindestens zwei Wochen vor der Jahreshauptversammlung zur Prüfung vorzulegen. 
    Der Befund ist der Jahreshauptversammlung vor der Entlastung bekanntzugeben.

    Die Kassenprüfer sind verpflichtet,
    ihre Prüfung auch auf die Satzungsmäßigkeit der Ausgaben und den rechtzeitigen und vollständigen Einzug der Einnahmen zu erstrecken.
    Sie haben insbesondere auch die Abrechnung von Veranstaltungen und etwaige Sonderkassen zu überwachen.
    Dazu können sie Prüfungen auch während des Geschäftsjahres und unverhofft vornehmen.

    \Clause{title={Wahlen und Abstimmungen}}
    \label{C:WahlenUndAbstimmungen}
    Für Wahlen in der Hauptversammlung wählt diese einen Wahlleiter und mind. 3 Wahlhelfer.

    Wahlen zum Vorstand finden geheim statt.
    Zum übrigen \DIFdelbegin \DIFdel{Ausschuß }\DIFdelend \DIFaddbegin \DIFadd{Ausschuss }\DIFaddend kann offen gewählt werden,
    wenn nur ein Bewerber vorgeschlagen ist und niemand widerspricht.
    Das gleiche gilt für die Kassenprüfer.

    Gewählt ist,
    wer mehr als die Hälfte der abgegebenen Stimmen erhält.
    Erreicht im ersten Wahlgang kein Bewerber die erforderliche Mehrheit,
    so findet ein zweiter Wahlgang statt,
    bei dem die relative Mehrheit entscheidet.
    Bei Stimmengleichheit entscheidet das Los.

    Beschlüsse werden durch Abstimmung mit einfacher Stimmenmehrheit gefaßt,
    soweit diese Satzung nichts anderes vorschreibt.
    Bei Stimmengleichheit ist die Abstimmung zu wiederholen.
    Ergibt sich wieder Stimmengleichheit,
    entscheidet die Stimme des Versammlungsleiters.

    Stimmenthaltungen bleiben unberücksichtigt.
    Betragen diese mehr als die Hälfte der Stimmberechtigten,
    ist die Abstimmung zu wiederholen.
    Ändern sich die Verhältnisse nicht,
    ist der Antrag abgelehnt.

    \Clause{title={Satzungsänderungen}}
    Die Änderung der Satzung bedarf der Zweidrittelmehrheit der anwesenden stimmberechtigten Mitglieder.

    Zur Änderung des Vereinszweckes ist die Zustimmung von mehr als der Hälfte aller stimmberechtigten Mitglieder erforderlich.

    \Clause{title={Auflösung des Vereins}}
    Bei Beratung und Beschluß über die Auflösung des Vereins muß mindestens die Hälfte aller stimmberechtigten Mitglieder anwesend sein.
    Der Beschluß bedarf der Mehrheit von drei Vierteln aller anwesenden Mitglieder.

    Ist weniger als die Hälfte der stimmberechtigten Mitglieder anwesend,
    so ist eine neue Versammlung einzuberufen,
    die ohne Rücksicht auf die Zahl der erschienenen Mitglieder beschlußfähig ist.

    Im Falle der Auflösung des Vereins wird das Vermögen nach Tilgung aller Schulden der Stadt Albstadt zur Verwaltung übertragen mit der Verpflichtung,
    es einem späteren Verein mit derselben Zielsetzung zu übergeben.
    Bildet sich innerhalb von fünf Jahren im Verwaltungsraum Albstadt kein solcher Verein,
    so fällt das Vermögen endgültig der Stadt Albstadt zu,
    die es ausschließlich und unmittelbar für gemeinnützige sportliche Zwecke zu verwenden hat.
    \DIFaddbegin 

    
    \Clause{title={Ordnungen}}
    \DIFadd{Zur Durchführung der Satzung kann sich der Verein folgende Ordnungen geben:
    }\begin{enumerate}[label=\alph*),noitemsep]
      \item \DIFadd{Geschäftsordnung
      }\item \DIFadd{Finanzordnung
      }\item \DIFadd{Beitragsordnung
      }\item \DIFadd{Datenschutzordnung
      }\item \DIFadd{Jugendordnung
      }\item \DIFadd{Ehrungsordnung
    }\end{enumerate}
    \DIFadd{sowie weitere, dem Verein dienliche Ordnungen.
    }

    \DIFadd{Die Mitgliederversammlung ist für den Erlass der Ordnungen zuständig.
    Ausgenommen davon sind die Geschäftsordnung,
    die vom Vorstand zu beschließen ist,
    sowie die Jugendordnung,
    die von der Vereinsjugend zu beschließen und vom Ausschuss zu bestätigen ist.
    }

    
    \Clause{title={Strafbestimmungen}}
    \DIFadd{Sämtliche Mitglieder des Vereins unterliegen der Ordnungsgewalt des Vereins.
    Der Vorstand kann gegen Mitglieder,
    die gegen die Satzung oder gegen Beschlüsse der Organe verstoßen oder das Ansehen,
    die Ehre und das Vermögen des Vereines schädigen,
    folgende Maßnahmen verhängen:
    }\begin{enumerate}[label=\alph*),noitemsep]
      \item \DIFadd{Rüge, Ermahnung, Verwarung, Verweis
      }\item \DIFadd{Zeitlich begrenztes Verbot der Teilnahme am Flugbetrieb und an Veranstaltungen des Vereines
      }\item \DIFadd{Geldstrafe bis zu € 250,00 je Einzelfall
      }\item \DIFadd{Ausschluss gemäß }\refL{C:ErloeschenDerMitgliedschaft}\DIFadd{~}\autoref{S:Ausschluss}
      \item \DIFadd{Zeitlich begrenztes Verbot des Betretens der vereinseigenen Anlagen und Einrichtungen
    }\end{enumerate}

    \Clause{title={Datenschutz}}
    \label{C:Datenschutz}
    \DIFadd{Unter Beachtung der gesetzlichen Vorgaben und Bestimmungen der EU-Datenschutz-Grundverordnung (DSGVO) und des Bundesdatenschutzgesetzes (BDSG) werden zur Erfüllung der Zwecke und Aufgaben des Vereins personenbezogene Daten über persönliche und sachliche Verhältnisse der Mitglieder des Vereins erhoben und in den vereinseigenen EDV-Systemen gespeichert, genutzt und verarbeitet. }\label{S:Datenschutz}

    \DIFadd{Als Mitglied des Württembergischen Landessportbundes e.V. (WLSB) ist der Verein verpflichtet, seine Mitglieder an den WLSB sowie an den jeweiligen Fachverband zu melden.
    }

    \DIFadd{Der Verein erlässt eine Datenschutzordnung, in der weitere Einzelheiten der Datenerhebung und der Datenverwendung sowie technische und organisatorische Maßnahmen zum Schutz der Daten aufgeführt sind. Die Datenschutzordnung wird auf Vorschlag der Vorstandschaft durch die Mitgliederversammlung beschlossen.
    }

    \DIFadd{Um die Aktualität der gemäß }\refPar{S:Datenschutz} \DIFadd{erfassten Daten zu gewährleisten,
    sind die Mitglieder verpflichtet, Veränderungen umgehend dem Verein mitzuteilen.
}

    \Clause{title={Ehrenrat}}
    \DIFadd{Der Ehrenrat besteht aus fünf Mitgliedern, die vom Ausschuss alle zwei Jahre gewählt werden.
}

    \DIFadd{Wählbar sind nur Mitglieder, die das vierzigste Lebensjahr vollendet haben, mindestens zehn Jahre als aktives Mitglied dem Verein und nicht dem Vorstand angehören.
}

    \DIFadd{Der Ehrenrat wählt seinen Vorsitzenden selbst.
}

    \DIFadd{Die Aufgaben des Ehrenrates sind:
}

    \begin{enumerate}[label=\alph*),noitemsep]
      \item \DIFadd{Unterbreitung von Ehrungsvorschlägen an den Vorstand gemäß den Bestimmungen der Ehrenordnung
      }\item \DIFadd{Unterbreitung von Änderungsvorschlägen zur Ehrenordnung an den Ausschuss
      }\item \DIFadd{Durchführung von Schlichtungs- und Ausschlussverfahren
    }\end{enumerate}

    \DIFadd{Der Ehrenrat tritt nur bei Bedarf zusammen.
    }

  \DIFaddend % \end{multicols}
\end{contract}

\end{document}